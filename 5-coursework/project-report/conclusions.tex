\begin{center}
    \addcontentsline{toc}{section}{Высновы}
    \section*{Высновы}    
\end{center}

Намі была апісаны тэарэтычны складнік ўсіх этапаў пабудовы разрэджанага воблака кропак на аснове набора выяваў; паказаныя адрозненні ў рэалізацыі агульнай задачы рэканструкцыі ад задачы з дадатковымі дадзенымі; прыведзеная магчымая рэалізацыя фармату файла з апісаннем дадзеных; праведзеная серыя эксперыментаў з мэтай вызначэння, ці дае прырост у хуткасці і якасці мадыфікаваная версія алгарытма.\par
На практыцы, выкарыстанне мадыфікаванай версіі алгарытма будзе азначаць наступнае: набор уваходных дадзеных змяшчае апроч выяваў файл з дадатковымі дадзенымі, якія мусяць быць папярэдне запісаныя ў яго ў адпаведным фармаце. Такім чынам у працэс апрацоўкі дадзеных пасля палёту дадаецца дадатковы этап, які патрабуе распрацоўкі дадатковага праграмнага забеспячэння.\par
Апроч варыяцыяў з ўваходнымі дадзенымі, магчымымі шляхам аптымізацыі працэса пабудовы мадэлі паверхні з'яўляюцца:
\begin{itemize}
    \item рэалізацыя алгарытма альбо ягоных крытычных частак на GPU замест CPU; 
    \item распаралельванне ўсіх этапаў рашэння задачы;
    \item іншыя спосабы аптымізацыі з выкарыстаннем асаблівасцяў апаратнага забеспячэння;
    \item рэалізацыя пабудовы спрошчанай і набліжанай да сапраўднай мадэлі паверхні, якая можа запускацца перад запускам паўнавартаснай рэканструкцыі і выкарыстоўвацца для папярэдняй ацэнкі карыстачом праграмы мясцовасці, масштабаў, узаемнага размяшчэння аб'ектаў на мадэлі.
\end{itemize}

Прапанаваныя вышэй падыходы складуць аснову маёй далейшай працы над задачай рэканструкцыі паверхні па здымках з беспілотных лятальных апаратаў.

\newpage