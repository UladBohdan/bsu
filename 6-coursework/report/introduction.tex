\begin{center}
    \addcontentsline{toc}{section}{УВОДЗІНЫ}
    \section*{УВОДЗІНЫ}
\end{center}

Рэканструкцыя паверхні зямлі па зыходных дадзеных з беспілотных лятальных апаратаў (БПЛА) - працэс,
які знайшоў практычнае прымяненне ў вялікай колькасці прыкладных задачаў. Такая задача можа ўзнікнуць
у працэсе будоўлі, пры ліквідацыі наступстваў кліматычных катастроф альбо ў сельскай гаспадарцы.
Развіццю і ўдасканаленню працэса рэканструкцыі паверхні паспрыяла адсутнасць у неабходнасці дарагой і
спецыялізаванай тэхнікі: для запуску алгарытмаў рэканструкцыі можа хапіць зыходных дадзеных у выглядзе
набора выяваў, зробленых звычайнай камерай на БПЛА якія, у сваю чаргу, з кожным годам становяцца
усё больш даступнымі шырокаму карыстачу.

Алгарытмы рэканструкцыі паверхні ўключаюць у сябе вялікую колькасць этапаў, у выніку выканання якіх
з набора выяваў і, магчыма, нейкіх дадатковых дадзеных, мы атрымліваем гатовую трохмерную мадэль паверхні,
якая можа выкарыстоўвацца ў прыкладных мэтах.

У гэтай працы мы падрабязна разгледзім некаторыя з этапаў пабудовы мадэлі, прывядзем вынікі эксперыментаў
з алгарытмамі вымання і апісання ключавых кропак, апішам распрацаваную праграму для пабудовы і візуалізацыі
трохмернай мадэлі па наборы выяваў (напрыклад, па дадзеных з БПЛА).

\newpage
