\begin{center}
    \addcontentsline{toc}{section}{ВЫСНОВЫ}
    \section*{ВЫСНОВЫ}
\end{center}

Мною была разгледжаная і засвоеная агульная тэорыя пабудовы трохмернай рэканструкцыі, а таксама шматлікія
тэарэтычныя аспекты задачаў, якія сустракаюцца ў галіне камп'ютарнага зроку.
Пры засваенні тэорыі я падрабязна спыніўся на этапе пошуку адпаведнасцяў паміж некалькімі выявамі і на этапе,
які яму папярэднічае - пошуку ключавых кропак і падліку адпаведных дэскрыптараў.

У практычнай частцы я правёў уласнае даследванне прымяняльнасці тых ці іншых дэтэктараў і дэскрыптараў,
зрабіў высновы наконт ўплыву характара выявы на атрыманы вынік.

Найбольшым унёскам у задачу рэканструкцыі стала рэалізацыя прыкладання, якое здзяйсняе будоўлю мадэлі з набора выяваў.
Праграма знаходзіцца ў актыўнай распрацоўцы, праца над ёй працягваецца. Код знаходзіцца ў вольным доступе
і можа быць знойдзены на GitHub.

Задача рэканструкцыі паверхні па дадзеных з БПЛА ўключае ў сябе мноства этапаў, кожны з якіх варты ўвагі і варты
асобнага даследвання. У будучым я планую працягваць распрацоўку тэмы, сачыць за сучаснымі падыходамі да працэса
рэканструкцыі, вылучаць асаблівасці задачы рэканструкцыі па дадзеных з БПЛА адносна агульнай задачай трохмернай
рэканструкцыі аб'ектаў.

\newpage
