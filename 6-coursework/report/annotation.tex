\begin{center}
    {\bf Анатацыя}
\end{center}

У курсавой працы апісаныя этапы пабудовы трохмернай мадэлі па дадзеных з БПЛА,
падрабязна разгледжаныя спосабы вымання ключавых кропак на выявах і метады іх
апісання, параўнаныя розныя тыпы дэтэктараў і дэскрыптараў, спосабы пошука адпаведнасцяў паміж кропкамі.
У працы прадстаўленыя вынікі эксперыментаў над выманнем ключавых кропак рознымі алгарытмамі,
прыведзенае апісанне праграмнага забеспячэння, над якім вялася распрацоўка, і якое
з'яўляецца канечным інструментам пабудовы трохмернай мадэлі.

\begin{center}
    {\bf Аннотация}
\end{center}

В курсовой работе описаны этапы построения трёхмерной модели по данным с БПЛА,
подробно рассмотрены способы извлечения ключевых точек на изображениях и методы их
описания, сравнены разные типы детекторов и дескрипторов, способы поиска соответствий между точками.
В работе представлены результаты экспериментов над извлечением ключевых точек разными алгоритмами,
приведено описание программного обеспечения, над которым велась разработка, и которое
является конечным инструментом построение трёхмерной модели.

\begin{center}
    {\bf Annotation}
\end{center}

The stages of the problem of three-dimensional surface reconstruction based on data from UAV
(unmanned aerial vehicle) are considered. The ways of detecting, extracting and describing
keypoints on the image are described in detail. The series of experiments determining the differences
in ways of extracting the keypoints were held. The progress of implementing the software for surface reconstruction from
the set of images is described.

\newpage
