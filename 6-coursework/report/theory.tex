\begin{center}
    \addcontentsline{toc}{section}{ТЭАРЭТЫЧНАЕ АБГРУНТАВАННЕ}
    \section*{ГЛАВА 1. \\ ТЭАРЭТЫЧНАЕ АБГРУНТАВАННЕ}
\end{center}

У агульным выпадку задачу рэканструкцыі можна падзяліць на некалькі этапаў наступным чынам:
\begin{itemize}
    \item Апрацоўка зыходных выяваў, пошук адпаведнасцяў паміж імі. Апрацоўка дадатковых
    зыходных дадзеных пры іх наяўнасці (калібровачныя параметры камеры, gps-трэк з БПЛА, інш.)
    \item Запуск алгарытма пошука размяшчэння камераў у прасторы. Пабудова разрэджанага воблака кропак
    праз мінімізацыю памылкі (фотаметрычнай, праэкцыі ці інш.)
    \item Ушчыльненне воблака кропак, нанясенне тэкстураў для атрымання сапраўднай шчыльнай трохмернай
    мадэлі паверхні зямлі альбо іншага аб'екта
\end{itemize}

Спынімся падрабязней на першым этапе.

\addcontentsline{toc}{subsection}{Простыя метады і метады, заснаваныя на ключавых кропках}
\subsection*{1.1 Простыя метады і метады, заснаваныя на ключавых кропках}

Метады, якія даюць яўленне пра пазіцыю камеры ў прасторы і будуюць першасную структуру сцэны, могуць быць
падзеленыя на два класы. Разгледзім іх падрабязней.

Першы - метады заснаваныя на \textit{пошуку асаблівасцяў} (англ. \textit{feature-based methods}).
Ідэя у выманні з кожнай асобнай выявы пэўнай колькасці асаблівасцяў (імі могуць быць, напрыклад, кропкі альбо лініі) і
пошук адпаведнасцяў паміж асаблівымі кропкамі на розных выявах (англ. \textit{matching}),
аднаўленне пазіцыяў камеры і структуры сцэны з дапамогай эпіпалярнай геаметрыі (геаметрыя стэрэабачання) і,
у завяршэнне, удакладненне параметраў праз мінімізацыю памылкі праэкцыі. Падобную працэдуру выконвае
вялікая колькасць алгарытмаў: даступнасць эфектыўных метадаў вымання асаблівасцяў і пошуку іх адпаведнасцяў
дазваляе рабіць гэта хутка і надзейна; разам з тым, такі падыход можа даваць недастатковую дакладнасць вынікаў,
быць прымяняльным толькі на вызначаным класе зыходных дадзеных (напрыклад, не дапускаць аднастайныя тэкстуры),
патрабуе надзейных алгарытмаў ацэнкі пазіцыі.

Другі клас метадаў заснаваны на гэтак званых \textit{простых метадах} (англ. \textit{direct methods}): у такіх
метадах непасрэдна выкарыстоўваюцца значэнні запісаныя ў пікселях, не адбываецца спробы выяўлення на выяве вызначанага
кшталту асаблівасцяў. Для апрацоўкі выкарыстоўваецца накірунак і велічыні лакальнага градыента інтэнсіўнасці.
Простыя метады, якія выкарыстоўваюць усю інфармацыю на выяве, нават у зонах з маленькім градыентам, апынуліся значна
эфектыўнейшымі за метады, заснаваныя на пошуку асаблівасцяў у сцэнах з нізкай тэкстураванасцю альбо ў выпадку
размытасці альбо дэфакусіроўкі \cite{direct-methods}.
Дастаткова працазатратнай працэдурай становіцца падлік фотаметрычнай памылкі (у параўнанні з падлікам памылкі праэкцыі).
Разам з тым, паколькі праца вядзецца непасрэдна са значэннямі пікселяў, час на пошук асаблівасцяў і падлік дэскрыптараў можа быць захаваны.

\addcontentsline{toc}{subsection}{Ключавыя кропкі і дэскрыптары}
\subsection*{1.2 Ключавыя кропкі і дэскрыптары}

Тут і далей мы будзем казаць пра ``ключавыя кропкі'' (англ. \textit{keypoints}), хаця ў тэрмінах таго ці іншага дэтэктара
ключавой кропкай можа называцца любая адзінка інфармацыі: кропка, акружнасць, лінія, кропка з накірункам і інш.

Дэтэктарам (англ. \textit{detector}) будзем называць алгарытм, які знаходзіць на выяве ключавыя кропкі.

Знойдзеная ключавая кропка пасля звычайна апісваецца ў нейкай вызначанай кароткай форме. Такое апісанне ключавой кропкі
будзем называць дэскрыптарам (англ. \textit{descriptor}). Часцей за ўсё сустракаюцца дэскрыптары ў выглядзе рэчаісных
вектароў альбо бінарных радкоў. Дэскрыптары дапамагаюць хутка знаходзіць адпаведныя адна адной ключавыя кропкі на
розных выявах. Знайсці адлегласць паміж двумя дэскрыптарамі і, такім чынам, вызначыць іхняе падабенства, з'яўляецца
асноўнай аперацыяй пры пошуку адпаведнасцяў паміж двумя выявамі, таму да надзейнасці і хуткасці падліку функцыі адлегласці
прад'яўляюцца асаблівыя патрабаванні. На практыцы адлегласць часцей за ўсё падліваецца праз Эўклідаву адлегласць L2 (для
рэчаісных вектароў) альбо праз адлегласць Хэмінга (для бінарных радкоў).

\vspace{5mm}

Ніжэй прывядзем кароткае апісанне найбольш распаўсюджаных дэтэктараў ключавых кропак і спосабаў іх апісання.

\addcontentsline{toc}{subsection}{Апісанні дэтэктараў і дэскрыптараў}
\subsection*{1.3 Апісанні дэтэктараў і дэскрыптараў}

Дэтэктар \textbf{FAST} (Features from Accelerated Segment Test) \cite{fast-paper}. Як вынікае з назвы, асноўнай перавагай дэтэктара
з'яўляецца ягоная хуткасць, гэта асабліва важна ў выпадку, калі дадзеныя паступаюць і патрабуюць апрацоўкі ў рэжыме
рэальнага часу і абмежаваных рэсурсаў. Найлепшым прыкладам будуць SLAM-прыкладанні (\textit{Simultaneous Localization and Mapping}),
у якіх у рэжыме рэальнага часу адбываецца ацэнка месцазнаходжання і пошук яго на мапе; SLAM-прыкладанні часта знаходзяць прымяненне
на мабільных прыладах, у тым ліку на БПЛА.

FAST з'яўляецца алгарытмам ``пошуку вуглоў'' (англ. \textit{corner detector}): кропка з'яўляецца ключавой, калі ў маленькіх выколіцах
пікселя алгарытму атрымліваецца распазнаць шаблон, іншымі словамі ``вугал''. Дадатковая аптымізацыя адбываецца з дапамогай метадаў машыннага
навучання.

Даследванні паказваюць, што ён у некалькі разоў хутчэйшы за іншыя існуючыя алгарытмы пошука ключавых кропак на аснове
вылучэння вуглоў, але, ў сваю чаргу ён недастаткова ўстойлівы да моцна зашумленых выяваў.

Дэскрыптар і дэтэктар \textbf{SIFT} \cite{sift-paper} быў прадстаўлены ў 2004 годзе і сёння з'яўляецца адным з найбольш распаўлсюджаных.
Важнымі характарыстыкамі алгарытмамі з'яўляецца ўстойлівасць да масштабавання і змянення арыентацыі выявы, яе афінных пераўтварэнняў
і зашумленых варыянтаў. Яскрава выраджаная характэрнасць ключавых кропак дазваляе паспяхова шукаць адпаведнасці паміж
рознымі выявамі. У \cite{sift-paper}, апроч апісання спосаба вымання ключавых кропак і фармата дэскрыптара, апісваюцца тэхнікі для
эфектыўнага пошука адпаведнасцяў паміж выявамі з SIFT-дэскрыптарамі, а таксама алгарытмы эфектыўнага распазнавання патэрнаў.
Такая завершанасць даследавання алгарытма дае яму вялікую перавагу перад іншымі і моцна паспрыяла ягонаму распаўсюду.
SIFT-дэскрыптарам з'яўляецца рэчаісны вектар размернасці 128.

Дэскрыптар і дэтэктар \textbf{SURF} \cite{surf-paper} быў упершыню апублікаваны ў 2006 годзе і пачаткова ідэя стварэння заключалася
ў стварэнне ``больш хуткага'' SIFT-а. Ён таксама з'яўляецца ўстойлівым да змены масштаба і паварота, можа быць паспяхова прыменены
для размытых выяваў; разам з тым, выкарыстанне SURF-а абмежаванае, калі мы маем справу са зменамі ў вугле назірання. Дэскрыптар SURF
можа быць як размернасці 64, так і 128 (як SIFT).

Звернем увагу на тое, што і SIFT і SURF з'яўляюцца запатэнтаванымі алгарытмамі і, негледзячы на вялікую колькасць рэалізацыяў з адкрытым
зыходным кодам, іх выкарыстанне ў камерцыйных прадуктах абмежаванае. У сваю чаргу, алгарытм ORB прапрыетарным не з'яўляецца.

Дэскрыптар і дэтэктар \textbf{ORB} (Oriented FAST and Rotated BRIEF) \cite{orb-paper} з'яўляецца своеасаблівай камбінацыяй
ключавых кропак FAST і дэскрыптара BRIEF. Мэтай стварэння ORB было пераўзысці SIFT у хуткасці (нават ахвяруючы надзейнасцю і дакладнасцю)
у сувязі з распаўсюдам мабільных маламагутных прыладаў. Дэскрыптары маюць выгляд бінарных радкоў.
Адпаведнасці паміж ORB дэскрыптарамі эфектыўна могуць быць знойдзеныя з дапамогай LSH (Locality-sensitive hashing).

Дэскрыптар \textbf{BRIEF} (Binary Robust Independent Elementary Features) \cite{brief-paper} стаў адным з першых дэскрыптараў,
якія апісываюць ключавыя кропкі
з дапамогай бінарных радкоў (SIFT альбо SURF, напрыклад, выкарыстоўваюць вектары з рэчаіснымі лікамі якія, у большасці выпадкаў,
пасля ўсё адно ў мэтах эфектыўнасці фарматуюцца ў бінарныя радкі, што дазваляе выкарыстоўваць эфектыўную адлегласць Хэмінга).

Дэскрыптар і дэтэктар \textbf{BRISK} (Binary Robust Invariant Scalable Keypoints) \cite{brisk-paper} ствараўся як
альтэрнатыва вышэйзгаданым SIFT i SURF, хуткасць працы якога дасягаецца праз выкарыстанне новага, заснаванага на FAST, дэтэктара
ў камбінацыі з бінарным дэскрыптарам.

Звернем увагу на тое, што некаторыя алгарытмы спалучаюць у сабе адразу як пошук ключавых кропак, так і падлік іх
дэскрыптараў. У агульным выпадку гэты працэс можа быць яўна падзелены: на мностве ключавых кропак атрыманых любым з дэтэктараў
можа быць запушчаны алгарытм, які для кожнай кропцы паставіць у адпаведнасць дэскрыптар вызначанага фармата.
На практыцы, некаторыя пары дэтэктар + дэскрыптар спалучаюцца добра і заўжды выкарыстоўваюцца разам, некаторыя сумяшчаюцца
вельмі кепска. Большасць алгарытмаў, якія знайшлі шырокае прымяненне, апісываюць адразу спосаб дэтэкцыі і апісання і ўтвараюць
такім чынам закрытую сістэму, раздзяленне якой не прыводзіць да добрых вынікаў.

\addcontentsline{toc}{subsection}{Спалучэнне простых метадаў і метадаў, заснаваных на ключавых кропках}
\subsection*{1.4 Спалучэнне простых метадаў і метадаў, заснаваных на ключавых кропках}

Вядомыя эфектыўныя алгарытмы, якія спалучаюць у сабе простыя метады і метады, заснаваныя на выманні асаблівасцяў.
Найбольшы распаўсюд такія алгарытмы знайшлі ў задачах апрацоўкі плыні дадзеных (відэаплыня, апрацоўка дадзеных у рэальным часе з БПЛА),
калі алгарытм будуе мапу мясцовасці і шукае сваё месцазнаходжанне ў рэальным часе.

Напрыклад, у \cite{svo} яўны пошук асаблівасцяў выклікаецца толькі калі новы кадр запускае ініцыялізацыю новых 3D кропак у прасторы, іначай
выкарыстоўваюцца простыя метады якія, у сваю чаргу, няяўна даюць адпаведнасці паміж кропкамі, пазначанымі раней у якасці асаблівых.

\addcontentsline{toc}{subsection}{Пошук адпаведнасцяў}
\subsection*{1.5 Пошук адпаведнасцяў (matching)}

Пошук асаблівых кропак і падлік адпаведных дэскрыптараў з'яўляецца, вядома ж, толькі падрыхтоўчым этапам да этапа параўнання дэскрыптараў з
дзвюх выяваў і пошук найлепшых параў (працэс матчынга, англ. \textit{matching}). Не існуе ўніверсальнага спосаба, які даваў бы найлепшыя
вынікі на любым тыпе асаблівых кропак і іх дэскрыптараў: для кожнага набора дэскрыптараў існуе найбольш эфектыўны спосаб пошуку
адпаведнасцяў. Таксама, выбар матчара залежыць ад пастаўленых намі задачаў і расстаўленых прыарытэтаў: ці нам найважнейшая
хуткасць, ці дакладнасць, ці дастаткова нам толькі $n$ найлепшых супадзенняў ці мы хочам атрымаць усе знойдзеныя.
Важным пытаннем з'яўляецца вызначэння парога, калі знойдзеная пара дэскрыптараў з'яўляецца заведама хібнай.

Самым простым спосабам знайсці адпаведнасці паміж дэскрыптарамі з дзвюх розных выяваў з'яўляецца просты пералік усіх магчымых параў, падлік нормы
(L2, Хэмінга, альбо любой іншай) і сартыроўка ўсіх параў па ўзрастанні значэнняў адлегласцяў (гэтак званы Brute-Force Matcher). Праз прастату
такі спосаб з'яўляецца простым у рэалізацыі, беспамылковым, але вельмі марудным. Практычнае прымяненне ва ўмовах апрацоўкі плыняў дадзеных альбо на
пост-апрацоўцы вялікіх наборах дадзеных, немагчымае.

Для выпадкаў, калі дакладнасцю можна ахвяраваць на карысць хуткасці (напрыклад, у працы з БПЛА хуткасць апрацоўкі
грае першасную ролю), былі вынайдзеныя і іншыя алгарытмы пошуку адпаведнасцяў.

Метады, пра якія ідзе гаворка, маюць агульную назву \textit{метадаў набліжанага пошуку бліжэйшых суседзяў}
(англ. \textit{Approximate Nearest Neighbors}). Падлічаныя намі дэскрыптары з'яўляюцца звычайным мноствам вектароў,
на якім зададзеныя суадносіны адлегласці. У практычнай частцы мы будзем выкарыстоўваць два віда такіх метадаў.
Для дэскрыптараў з рэчаіснымі вектарамі (SIFT, SURF) гэта будзе метад заснаваны на k-мерных дрэвах
(англ. \textit{k-dimensional trees}), для бінарных дэскрыптараў - Locality-sensitive Hashing (LSH) - імавернасны
метад паніжэння размернасці дадзеных.

\newpage
