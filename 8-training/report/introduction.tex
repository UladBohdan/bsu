\begin{center}
    \addcontentsline{toc}{section}{УВОДЗІНЫ}
    \section*{УВОДЗІНЫ}
\end{center}

Задача рэканструкцыі паверхні зямлі па дадзеных з беспілотных лятальных апаратаў
(БПЛА) узнікае ўсё ў большай колькасці сфераў жыццядзейнасці: ад сельскай гаспадаркі да
ацэнкі наступстваў прыродных катастрофаў. Патрэбнасць у эфектыўным рашэнні
задачы таксама звязаная з шырокай даступнасцю БПЛА і з распаўсюдам танных камераў.

Задача рэканструкцыі цесна звязаная са шматлікімі сумежнымі задачамі:
распазнаванне аб'ектаў, навігацыя ў прасторы, пабудова мапы мясцовасці.
Асаблівую цікаўнасць прадстаўляюць рашэнні, якія выконваюцца ў рэальным часе;
праца ў рэальным часе для алгарытмаў навігацыі і абхода перашкодаў можа быць крытычнай для
аўтаномных БПЛА, у адрозненні ад наземных робатаў, якія могуць на нейкі час спыніцца і
дачакацца пабудовы маршрута.

Падчас праходжання пераддыпломнай практыкі я працягнуў даследаваць тэорыю ў галіне
трохмернай рэканструкцыі, шчыльна пазнаёміўся з канцэпцыяй SLAM (Simultaneous
Localization and Mapping) алгарытмаў, а таксама часткова рэалізаваў сістэму, якая
для эфектыўнасці рэканструкцыі спалучае ў сабе SLAM-частку і больш
традыцыйныя метады рэканструкцыі паверхні.

\newpage
