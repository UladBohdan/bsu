\begin{center}
    \addcontentsline{toc}{section}{ВЫСНОВЫ}
    \section*{ВЫСНОВЫ}
\end{center}

У справаздачы былі апісаныя асноўныя паняцці, датычныя да задачы трохмернай рэканструкцыі
паверхні па дадзеных з БПЛА, сфармуляваная канцэпцыя SLAM, апісаныя некаторыя
найбольш вядомыя і паспяхова рэалізаваныя SLAM сістэмы.

Негледзячы на тое, што SLAM, рэалізаваны для працы ў рэальным часе, і часцей за ўсё выкарыстоўваемы
для навігацыі ў прасторы, кепска спраўляецца з задачай пабудовы шчыльнай трохмернай
мадэлі, вынікі адпрацоўкі SLAM могуць быць вельмі карыснымі ў якасці апрыорных
дадзеных для запуску афлайн алгарытмаў.

У якасці практычнай рэалізацыі я распрацаваў дызайн і часткова рэалізаваў сістэму, якая б
займалася эфектыўнай рэканструкцыяй паверхні па дадзеных з беспілотных лятальных апаратаў, у аснове якой ляжыць
два канкуруючых падыхода: SLAM-алгарытм спалучаны разам з класічным падыходам да
рэканструкцыі. Надалей я працягну ўдасканальваць кожны асобны элемент гэтай сістэмы,
а таксама буду паляпшаць камунікацыі паміж рознымі часткамі сістэмы, каб сістэма
ўспрымалася адзіным суцэльным механізмам.

\newpage
