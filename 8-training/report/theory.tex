\begin{center}
    \addcontentsline{toc}{section}{ГЛАВА 1. ЗАДАЧА РЭКАНСТРУКЦЫІ ПАВЕРХНІ}
    \section*{ГЛАВА 1. \\ Задача рэканструкцыі паверхні}
\end{center}

\addcontentsline{toc}{subsection}{1.1 Агульныя звесткі}
\subsection*{1.1 Агульныя звесткі}

У агульным выпадку задача рэканструкцыі паверхні фармулюецца наступным чынам:
неабходна рэканструяваць трохмерны аб'ект па мностве здымкаў зробленых з розных ракурсаў.
Задача досыць натуральная і калі чалавеку дастаткова кінуць
позірк на аб'ект, каб уявіць ягоную трохмерную структуру, алгарытмічна задача
ўсё яшчэ застаецца не цалкам вырашанай, патрабуе досыць вялікіх вылічальных магутнасцяў
і не працуе ўніверсальна добра для любых асяроддзяў і любых умоваў здымак.

Задача рэканструкцыі можа таксама фармулявацца і для іншых тыпаў уваходных дадзеных,
для дадзеных з іншых датчыкаў, такіх як акселерометр, гіраскоп ці GPS-датчык;
замест манакулярнай камеры можа прымяняцца RGB-D (вяртае дадатковы слой глыбіняў)
ці стэрэакамера, якая ў пэўным сэнсе імітуе бінакулярны чалавечы зрок.

У залежнасці ад таго, якія выходныя дадзеныя нас цікавяць, алгарытм можа будаваць
воблака кропак рознай шчыльнасці альбо будаваць шчыльную тэкстураваную мадэль,
максімальна падобную на рэальны трохмерны аб'ект. Розныя патрабаванні да выходных
дадзеных патрабуюць адрозных алгарытмічных рашэнняў і наяўнасці розных вылічальных магутнасцяў.

Варта дадаць, што даследаванні ў гэтай галіне кампутарнага зроку развіваюцца
таксама праз удасканаленне апаратнага забеспячэння: падыходы, якія некалькі год
таму былі практычна нерэалізуемымі, з развіццём тэхналогіяў даюць ім новае жыццё.

\addcontentsline{toc}{subsection}{1.2 Сферы прымянення}
\subsection*{1.2 Сферы прымянення}

Цікавасць задачы рэканструкцыі таксама ў запатрабаванні атрымання рашэння ў абсалютна розных сферах
жыцця, у кожнай на сваімі асаблівасцямі.
БПЛА выкарыстоўваюцца надзвычайнымі службамі для ацэнкі наступстваў прыродных катастроф,
у сельскай гаспадарцы, дарожнымі службамі. Патрабаванні да хуткасці працы алгарытмаў рэканструкцыі
натуральныя - хуткасць працы ў некаторых галінах жыцця крытычная і разбор вялікіх аб'ёмаў
неапрацаваных дадзеных чалавекам можа быць немагчымы. Патрабаванні да працы алгарытмаў
рэканструкцыі ў рэальным часе ў большасці выпадкаў з'яўляюцца пры навігацыі і
аўтаномным руху, у такім выпадку шчыльная рэканструкцыя можа быць залішняй і
разрэджаная мадэль у выглядзе воблака кропак цалкам задаволіць патрабаванні.

\newpage
