\documentclass[12pt,a4paper]{extarticle}
\usepackage[utf8]{inputenc}
\usepackage[left=3cm,right=1.5cm,
top=2cm,bottom=2cm,bindingoffset=0cm]{geometry}
\usepackage[english,russian]{babel}
\usepackage[pdftex]{graphicx}
\usepackage{amsfonts}
\usepackage{amsmath}
\usepackage{verbatim}

\tolerance=1
\emergencystretch=\maxdimen
\hyphenpenalty=10000
\hbadness=10000

\pagenumbering{gobble}
\begin{document}
	\begin{center}
			БЕЛАРУСКІ ДЗЯРЖАЎНЫ ЎНІВЕРСІТЭТ \\
			Факультэт прыкладной матэматыкі і інфарматыкі \\
			Кафедра дыскрэтнай матэматыкі і алгарытмікі \\
	\end{center}
	\vspace{30pt}
	\begin{flushright}
		\textbf{Зацвярджаю} \\
		Загадчык кафедры \underline{\hspace{150pt}} \\
		``\underline{\hspace{25pt}}'' \underline{\hspace{70pt}} 20 \underline{\hspace{25pt}} \\
	\end{flushright}
	\vspace{30pt}
	\begin{center}
		\textbf{
			Заданне на практыку \\
			па спецыяльнасці ``інфарматыка'' \\
		}
	\end{center}
	\vspace{10pt}
	Студэнту \underline{\hspace{120pt}\textit{Богдану Уладзіславу}\hspace{120pt}} \\
	\renewcommand{\labelenumii}{\arabic{enumi}.\arabic{enumii}.}
	\begin{enumerate}
		\item Тэма практыкі: \underline{\textit{Рэканструкцыя паверхні па дадзеных з беспілотных}} \\
		\underline{\textit{лятальных апаратаў}}
		\item Спіс рэкамендаванай літэратуры:
			\begin{itemize}
				\item А.В. Тузиков, С.А Шейнин, Д.В. Жук. Математическая морфология, моменты,
				стереоoбработка: избранные вопросы обработки и анализа цифровых изображений. Минск, Белорус. наука, 2006.-198с
				\item Jiaxin Li, Yingcai Bi, Menglu Lan, Hailong Qin, Mo Shan, Feng Lin, and Ben M. Chen.
				Real-time simultaneous localization and mapping for uav: A survey. 2016.
				\item Cesar Cadena, Luca Carlone, Henry Carrillo, Yasir Latif, Davide Scaramuzza,
				Jose Neira, Ian D. Reid, and John J. Leonard. Simultaneous localization and mapping:
				Present, future, and the robust-perception age. CoRR, abs/1606.05830, 2016.
				\item Morgan Quigley, Ken Conley, Brian P. Gerkey, Josh Faust, Tully Foote, Jeremy Leibs,
				Rob Wheeler, and Andrew Y. Ng. Ros: an open-source robot operating system.
				In ICRA Workshop on Open Source Software, 2009.
			\end{itemize}
		\item Пералік пытанняў, якія падлягаюць распрацоўцы альбо кароткі змест разлікова-тлумачальнай нататкі:
			\begin{itemize}
				\item вывучэнне рэкамендаванай літаратуры;
				\item агляд існуючых SLAM-рэалізацыяў, правядзенне эксперыментаў над імі, ацэнка магчымасцяў
				па выманні дадатковых дадзеных з алгарытмаў;
				\item распрацоўка ўласнага праграмнага забеспячэння на аснове існуючых рэалізацыяў
				і бібліятэк;
				\item ацэнка эфектыўнасці працы, агляд шляхоў аптымізацыі;
				\item падрыхтоўка справаздачы.
			\end{itemize}
		\newpage
		\item Прыкладны каляндарны графік:
			\begin{itemize}
				\item \textbf{люты (1-2 тыдні)} - азнаямленне з тэарэтычным матэрыялам і літаратурай па тэме,
				агляд разнастайных рэалізацыяў SLAM-алгарытмаў
				\item \textbf{люты (3 тыдзень)} - ацэнка магчымасцяў спалучэння аффлайн алгарытмаў рэканструкцыі
				з SLAM-падыходамі
				\item \textbf{сакавік (4-5 тыдні)} - распрацоўка праграмнага забеспячэння, якое б спалучала ў сабе
				папярэдні запуск SLAM-алгарытма з паслядоўнай выгрузкай дадзеных для правядзення
				паўнавартаснай шчыльнай рэканструкцыі паверхні
				\item \textbf{сакавік (6 тыдзень)} - падсумаванне вынікаў,
				афармленне тэарэтычнай і практычнай частак справаздачы
			\end{itemize}
		\item Кіраўнікі практыкі: \\
		ад прадпрыемства \underline{Тузікаў А. В., генеральны дырэктар} \\
		ад кафедры \underline{Жылка А. І., асістэнт} \\
		\item Дата выдачы задання: \underline{5 лютага 2018 г.}
		\item Тэрмін здачы справаздачы: \underline{22 сакавіка 2018 г.}
	\end{enumerate}
	\vspace{30pt}
	Кіраўнік $\underset{\text{(подпіс, дата)}}{\underline{\hspace{150pt}}} \hspace{5pt}  /\underline{\hspace{5pt}\textit{Жылка А.І.}\hspace{5pt}}/$ \\
	\vspace{15pt} \\
	Подпіс студэнта $\underset{\text{(подпіс, дата)}}{\underline{\hspace{150pt}}}$
\end{document}
