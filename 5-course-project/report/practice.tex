\begin{center}
    \addcontentsline{toc}{section}{Практычная рэалізацыя}
    \section*{Практычная рэалізацыя}    
\end{center}

Практычная рэалізацыя, у маім выпадку, складалася з напісання невялікіх праграмаў для адпрацоўкі асобных этапаў рэканструкцыі (такіх як пошук ключавых кропак), а таксама унясення зменаў і эксперыменты над ужо існуючымі распрацоўкамі ў галіне кампутарнага зроку і задачы рэканструкцыі дадзеных, атрыманых з беспілотных лятальных апаратаў.\par

\vspace{5mm}

Мэтай курсавога праэкта не была рэалізацыя алгарытмаў пошуку ключавых кропак і пошуку адпаведнасцяў паміж імі. Гэтыя алгарытмы рэалізаваныя шматразова у разнастайных выглядах і ў складзе розных бібліятэк - напрыклад, OpenCV - найбольш распаўсюджанай бібліятэкай, якая рэалізуе алгарытмы кампутарнага зроку.\par

\vspace{5mm}

\addcontentsline{toc}{subsection}{Рэалізацыя фармата файла дадатковых дадзеных}
\subsection*{Рэалізацыя фармата файла дадатковых дадзеных}
Дадзеныя пра адпаведнасці паміж кропкамі на розных выявах мы атрымалі ў выніку аперацый feature extraction і feature matching, прыклады рэалізацыі якіх прыведзеныя ў папярэдніх пунктах.\par
Вызначым фармат файла, які будзем перадаваць праграме разам з наборам выяваў і які будзе структуравана захоўваць дадзеныя з іншых датчыкаў на беспілотным лятальным апараце (вонкавы GPS-датчык, акселерометр і інш.).\par
Патрабаванні да фармата файла будуць наступныя:
\begin{itemize}
    \item У файле мусіць быць апісаная фізічная камера, пры дапамозе якой быў падрыхтаваны адпаведны набор выяваў. Камера задаецца з дапамогай пяці параметраў: фокуснай адлегласці, зрушэнні цэнтра праэкцыі адносна кропкі праламлення і два каэфіцыэнта радыяльнай дэфармацыі. Маем радок параметраў \verb|f, cx, cy, k1, k2|. Нагадаем, што унутраныя параметры камеры агульныя для кожнай "віртуальнай камеры", бо мяркуецца, што ўвесь набор дадзеных падрыхтаваны адной фізічнай камерай.
    \item Наступным радком апісываем колькасць выяваў у наборы дадзеных. Маем параметр \verb|num_images|.
    \item Далей будзе ісці апісанне кожнай выявы з выкарыстаннем яе імя, кватэрніона \verb|q|, якому адпавядае вектар даўжыні 4 і які апісвае паварот камеры ў прасторы, а таксама вектара \verb|t| даўжыні 3, які апісвае зрушэнне камеры ў прасторы адносна вызначанай кропкі. Вектар \verb|t| можа з'яўляецца геаграфічнымі каардынатамі кропкі ў прасторы, альбо апісываць адноснае становішча кропак іншым чынам.
\end{itemize}

Маем наступны фармат файла:
\begin{verbatim}
    <f> <cx> <cy> <k1> <k2>
    <num_images>
    <image_name_0> <4d q vector> <3d t vector (lat, lon, alt)>
    ...
    <image_name_num_images> ...
\end{verbatim}

Ніжэй прыведзены прыклад файла:
\begin{verbatim}
    2775.27 1600.0 1200.0 0.1 0.1
    4
    IMG_1190.JPG 0.093 -0.011 0.167 0.981 0.671 -1.171 0.316
    IMG_1191.JPG 0.017 -0.019 0.122 0.992 0.611 0.301 -0.201
    IMG_1192.JPG -0.058 0.015 0.085 0.994 0.062 1.951 -0.136
    IMG_1193.JPG 0.058 -0.082 -0.009 0.994 1.447 3.954 0.290
\end{verbatim}

Што ж тычыцца праграмы, то ў ёй быў рэалізаваны спецыяльны клас з метадамі чытання і апрацоўкі дадзеных з файла. Аб'ект такога класа уключаецца ў алгарытм і прадастаўляе просты доступ да дадзеных апісаных вышэй. Апроч іншага, гэты клас суадносіць выявы, апісаныя ў файле з выявамі з базы дадзеных, робіць высновы, наколькі поўна апісаны файл у параўнанні са змесцівам базы.

\addcontentsline{toc}{subsection}{Кантроль над алгарытмам}
\subsection*{Кантроль над алгарытмам}
Найбольшую цікаўнасць для нас прадстаўляе ўплыў наяўнасці дадатковых дадзеных на хуткасць працы алгарытма і на якасць атрыманай мадэлі. Сутнасць запуску серыі эксперыментаў заключалася ў запуску алгарытма з рознымі канфігурацыямі і параўнанні вынікаў. У рэалізацыю алгарытма рэканструкцыі я дадаў некалькі дадатковых сцягоў (індыкатараў), які задаюць канфігурацыю алгарытма. Такім чынам канфігураваць алгарытм стала магчымым праз наступныя параметры:
\begin{itemize}
    \item \verb|use_drone_data| - вызначае, ці выкарыстоўваць (ці чытаць наогул) дадзеныя з вызначанага намі вышэй файла.
    \item \verb|use_qvec_tvec_estimations| - вызначае, ці выкарыстоўваць у алгарытме дадзеныя, якія апісываюць паварот і зрушэнне камеры для кожнай асобна ўзятай выявы. Такім чынам, гэты сцяг указывае, ці выкарыстоўваць усе дадзеныя, ці абмежавацца унутранымі параметрамі камеры. Ігнаруецца, калі не пастаўлены \verb|use_drone_data|.
    \item \verb|refine_focal_length| - калі выстаўлены, то пры запуску алгарытма фокусная адлегласць апраксімуецца вызначанымі намі значэннямі, але ў ходзе ітэратыўнага выканання алгарытма значэнні могуць мяняцца для мінімізацыі памылкі. У процілеглым выпадку значэнні фіксаваныя, і фокусная адлегласць выступае ў якасці канстанты, а не параметра, які можа быць удакладнены ў працэсе выканання алгарытма.
    \item \verb|refine_principal_point| - гэты параметр выконвае ролю падобную да той, што адведзена \verb|refine_focal_length|, але ў сваю чаргу адказвае за згаданыя вышэй унутраныя параметры \verb|cx, cy| камеры. Звычайны гэта сцяг выстаўляецца на \verb|false|, бо для большасці камер параметры \verb|cx, cy| могуць быць дакладна вызначаныя як cx = (шырыня выявы) / 2, cy = (вышыня выявы) / 2 (задаюцца ў пікселях).
    \item \verb|refine_extra_params| - адказвае за паводзіны адносна параметраў радыяльнай дэфармацыі: алгарытм спрабуе палепшыць іх, калі сцяг выстаўлены, і пакідае пастаяннымі ў адваротным выпадку.
\end{itemize}

Варыруя значэнні сцягоў, вызначаных вышэй, будзем атрымліваецца розныя паводзіны алгарытма. Гэтыя паводзіны мы і будзем аналізаваць.\par

\addcontentsline{toc}{subsection}{Серыя эскперыментаў}
\subsection*{Серыя эскперыментаў}
Эксперымент будзе прадстаўляць сабой набор дадзеных, на якім будуць запускацца розныя канфігурацыі алгарытма. Вынікамі запуску алгарытма будзем лічыць час, за які ён адпрацоўвае, а таксама якасць пабудаванай мадэлі. Якасць мадэлі будзем ацэньваць колькасцю кропак, якія ў выніку былі ў яе дададзеныя, наяўнасцю ці адсутнасцю выкідаў, шчыльнасцю воблака кропак, а таксама досыць суб'ектыўнымі ацэнкамі, такімі як вонкавы выгляд мадэлі і вонкавае падабенства да мадэляў, пабудаваных з іншымі канфігурацыямі алгарытма.\par
У абодвух апісаных ніжэй эксперыментах выкарыстоўваецца аднолькавы набор выяваў і адпаведных ім параметраў - розніца ў памерах набораў. Усе запускі адбываліся на камп'ютары з аднолькавай канфігурацыяй, нагрузка ад знешніх працэсаў на працэсар была сведзеная да мінімума.

\subsubsection*{Эксперымент на маленькім наборы дадзеных}
У гэтым эксперыменце ў якасці дадзеных былі ўзятыя {\bf 16} выяваў са згаданага вышэй набора. Пошук ключавых кропак адпрацаваў за {\bf 0.79 хвіліны}, пошук адпаведнасцяў - за {\bf 1.19 хвіліны}. Вынікі запускаў прыведзеныя ў табліцы 1.
\begin{table}[h!]
\centering
\begin{tabular}{ |m{1em}|m{2em}|m{2em}|m{2em}|m{2em}|m{2.5em}|m{3.2em}|m{12em}| }
 \hline
 & {\small use drone data} & {\small use qvec tvec estim. } & {\small refine focal length} & {\small refine princ. point} & {\small refine extra params} & Час (хвілін) & Вынікі \\
 \hline
 1 & 0 & 0 & 1 & 0 & 1 & 0.52 & Стандартны запуск алгарытма без выкарыстання вонкавых дадзеных. Атрымалі мадэль на 8.6k кропак. \\
 \hline
 2 & 1 & 0 & 0 & 0 & 1 & 0.45 & Аб'яўленне фокуснай адлегласці канстантнай велічынёй. Пабудавалася мадэль на ~4k кропак. Заўважная параза ў якасці і невялікае паскарэнне. \\ 
 \hline
 3 & 1 & 1 & 0 & 0 & 1 & 0.29 & Аб'яўленне ўсіх вонкавых дадзеных канстантнымі. Мадэль на 5.6k кропак з візуальна параўнальнай якасцю і значным прыростам у часе. \\ 
 \hline
\end{tabular}
\captionsetup{labelformat=empty}
\caption{Табліца 1: вынікі эксперымента на малым наборы дадзеных}
\end{table}

\subsubsection*{Эксперымент на вялікім наборы дадзеных}
\textbf{127} выяваў (пашыраная версія таго ж набора). Пошук ключавых кропак за \textbf{2.11} хвіліны і пошук адпаведнасцяў за \textbf{43.59} хвілінаў. Вынікі - у табліцы 2.
\begin{table}[h!]
\centering
\begin{tabular}{ |m{1em}|m{2em}|m{2em}|m{2em}|m{2em}|m{2.5em}|m{3.2em}|m{12em}| }
 \hline
 & {\small use drone data} & {\small use qvec tvec estim. } & {\small refine focal length} & {\small refine princ. point} & {\small refine extra params} & Час (хвілін) & Вынікі \\
 \hline
 1 & 0 & 0 & 1 & 0 & 1 & 12.54 & Запуск алгарытма са стандартнай канфігурацыяй. Атрыманая мадэль на 59k кропак. \\
 \hline
\end{tabular}
\captionsetup{labelformat=empty}
\caption{Табліца 2: вынікі эксперымента на вялікім наборы дадзеных}
\end{table}

\begin{table}[h!]
\centering
\begin{tabular}{ |m{1em}|m{2em}|m{2em}|m{2em}|m{2em}|m{2.5em}|m{3.2em}|m{12em}| }
 \hline
 2 & 1 & 0 & 0 & 0 & 1 & 26.22 & Жорсткае фіксаванне фокуснай адлегласці выглядае кепскай ідэяй. Невытлумачальны рост часу і значнае пагаршэнне якасці мадэлі (22.8k кропак, візуальна на парадак горш за першы запуск). \\ 
 \hline
 3 & 1 & 1 & 0 & 0 & 1 & 7.68 & Выкарыстанне ўсіх магчымых вонкавых дадзеных як пастаянных. Атрыманая мадэль на 46k кропак. Вонкава горш за першы запуск, але праз значна прырост па часе можа выступаць у якасці хуткай пабудовы "накіда" мадэлі. \\ 
 \hline
\end{tabular}
\captionsetup{labelformat=empty}
\caption{Працяг табліцы 2}
\end{table}

\addcontentsline{toc}{subsection}{Іншыя спосабы ацэнкі вынікаў}
\subsection*{Іншыя спосабы ацэнкі вынікаў}

Ацэнка вынікаў эксперыментаў - найважнейшая ягоная частка. У папярэднім пункце мы рабілі высновы, абапіраючыся ў асноўным на вонкавае параўнанне мадэляў, што не з'яўляецца строгай ацэнкай вынікаў працы. Для строгай і акуратнай ацэнкі вынікаў можна выкарыстаць наборы дадзеных, якія былі распрацаваныя спецыяльна для тэставання і ацэнкі эфектыўнасці алгарытмаў рэканструкцыі. Прыклад рэсурса з такімі наборамі дадзеных - па спасылцы \underline{https://www.sensefly.com/drones/example-datasets.html}. Наборы дадзеных ідуць адразу разам з апісаннем вынікаў працы на іншых рэалізацыях алгарытмаў. Тэставанне і эксперэментаванне з дапамогай такіх набораў уваходзіць у набліжэйшыя планы па працы над праэктам.

\newpage