\documentclass[12pt,a4paper]{extarticle}
\usepackage[utf8]{inputenc}
\usepackage[left=3cm,right=1.5cm,
top=2cm,bottom=2cm,bindingoffset=0cm]{geometry}
\usepackage[english,russian]{babel}
\usepackage[pdftex]{graphicx}
\usepackage{amsfonts}
\usepackage{amsmath}
\usepackage{verbatim}

\pagenumbering{gobble}
\begin{document}
	\begin{center}
			БЕЛАРУСКІ ДЗЯРЖАЎНЫ ЎНІВЕРСІТЭТ \\
			Факультэт прыкладной матэматыкі і інфарматыкі \\
			Кафедра дыскрэтнай матэматыкі і алгарытмікі \\
	\end{center}
	\vspace{30pt}
	\begin{center}
		\textbf{
			ЗАДАННЕ \\
			ПА ПАДРЫХТОЎЦЫ КУРСАВОГА ПРАЭКТА \\
		}
	\end{center}
	\vspace{10pt}
	Студэнту \underline{\hspace{5pt}\textit{3}\hspace{5pt}} курса \underline{\hspace{120pt}\textit{Богдану Уладзіславу}\hspace{120pt}} \\
	\renewcommand{\labelenumii}{\arabic{enumi}.\arabic{enumii}.}
	\begin{enumerate}
		\item Тэма праэкта: \underline{\textit{Рэканструкцыя паверхні па дадзеных з беспілотных лятальных}} \\
		\underline{\textit{апаратаў}}
		\item Тэрмін здачы студэнтам скончанага праэкта: \underline{\textit{снежань 2016}}
		\item Змест курсавога праэкта:
		\begin{enumerate}
			\item \textit{Даследванне асноўных этапаў задачы рэканструкцыі паверхні}
			\item \textit{Падрыхтоўка кодавай базы для наступнай рэалізацыі алгарытмаў}
			\item \textit{Вывучэнне алгарытмаў пабудовы воблака кропак на аснове здымкаў і іншых дадзеных з беспілотных лятальных апаратаў; даследванне спосабаў іх аптымізацыі}
			\item \textit{Распрацоўка спосаба перадачы алгарытму дадатковых дадзеных, атрыманых з БПЛА}
			\item \textit{Правядзенне серыі эксперыментаў для параўнання вынікаў працы розных мадыфікацый алгарытму}
		\end{enumerate}
	\end{enumerate} 
	\vspace{30pt}
	Кіраўнік курсавога праэкта: $\underset{\text{(подпіс, дата)}}{\underline{\hspace{150pt}}} \hspace{5pt}  /\underline{\hspace{5pt}\textit{Емельянаў І. А.}\hspace{5pt}}/$ \\
	\vspace{15pt} \\
	Заданне прыняў да выканання: $\underset{\text{(подпіс, дата)}}{\underline{\hspace{150pt}}}$
\end{document}