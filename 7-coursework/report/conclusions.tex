\begin{center}
    \addcontentsline{toc}{section}{ВЫСНОВЫ}
    \section*{ВЫСНОВЫ}
\end{center}

У працы былі рагледжаныя асноўныя паняцці, датычныя да задачы трохмернай рэканструкцыі
паверхні па дадзеных з БПЛА, сфармуляваная канцэпцыя SLAM, апісаныя некаторыя
найбольш вядомыя і паспяхова рэалізаваныя SLAM сістэмы.

Негледзячы на тое, што SLAM, рэалізаваны для працы ў рэальным часе, і часцей за ўсё выкарыстоўваемы
для навігацыі ў прасторы, кепска спраўляецца з задачай пабудовы шчыльнай трохмернай
мадэлі, вынікі адпрацоўкі SLAM могуць быць вельмі карыснымі ў якасці апрыорных
дадзеных для запуску афлайн алгарытмаў.

У працы таксама прыводзіцца апісанне фрэймворка ROS, які выкарыстоўваецца для
запуску большасці SLAM рэалізацыяў, а таксама прыведзеныя вынікі правядзення
эксперыментаў па даследванні збежнасці ORB-SLAM.

\newpage
