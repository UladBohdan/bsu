\begin{center}
    {\bf Анатацыя}
\end{center}

У курсавым праекце апісваецца задача рэканструкцыі паверхні па дадзеных з беспілотных
лятальных апаратаў (БПЛА), прыводзіцца апісанне канцэпта SLAM-сістэм (адначасовай
лакалізацыі і пошуку на мапе), разглядаюцца некаторыя найбольш паспяховыя SLAM-алгарытмы.
У працы прадстаўленыя вынікі эксперыментаў па даследванні збежнасці SLAM-сістэм і
апісваюцца магчымасці інтэграцыі з класічным падыходам да рэканструкцыі.

\begin{center}
    {\bf Аннотация}
\end{center}

В курсовом проекте описывается задача реконструкции поверхности по данным с беспилотных
летательных аппаратов (БПЛА), приводится описание концепта SLAM-систем (одновременной
локализации и поиска на карте), рассматриваются некоторые наиболее успешные SLAM-алгоритмы.
В работе представлены результаты экспериментов по исследованию сходимости SLAM-систем и
описываются возможности интеграции с классическим подходом к реконструкции.

\begin{center}
    {\bf Annotation}
\end{center}

The surface reconstruction based on data from unmanned aerial vehicles (UAV's) problem is described
as well as the concept of SLAM (simultaneous localization and mapping) and some most
successful SLAM implementations.
The work contains results of the experiments determining convergence abilities of
SLAM systems. Ways to integrate SLAM with classical approach are considered.


\newpage
