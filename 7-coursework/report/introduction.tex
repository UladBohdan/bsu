\begin{center}
    \addcontentsline{toc}{section}{УВОДЗІНЫ}
    \section*{УВОДЗІНЫ}
\end{center}

Задача рэканструкцыі паверхні зямлі па дадзеных з беспілотных лятальных апаратаў
(БПЛА) узнікае ўсё ў большай колькасці сфераў жыцця: ад сельскай гаспадаркі да
ацэнкі наступстваў прыродных катастрофаў. Патрэбнасць у эфектыўным рашэнні
задачы таксама звязаная з шырокай даступнасцю БПЛА і камер.

Непасрэдна задача рэканструкцыі цесна звязаная са шматлікімі сумежнымі задачамі:
распазнаванне аб'ектаў, навігацыя ў прасторы, пабудова мапы мясцовасці.
Асаблівую цікаўнасць прадстаўляюць рашэнні, якія выконваюцца ў рэальным часе;
праца ў рэальным часе для алгарытмаў навігацыі і абхода перашкодаў можа быць крытычнай для
аўтаномных БПЛА, у адрозненні ад наземных робатаў, якія могуць на нейкі час спыніцца і
дачакацца пабудовы маршрута.

У гэтай працы я падсумую найважнейшыя аспекты, якія ўзнікаюць пры рашэнні задачы
рэканструкцыі паверхні, а таксама разгледжу шэраг алгарытмаў з агульнай назвай
SLAM (Simultaneous Localization and Mapping), якія могуць быць эфектыўна прымененыя
да задачаў, якія патрабуюць найбольшай хуткасці выканання, такіх як навігацыя.

\newpage
