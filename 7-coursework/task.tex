\documentclass[12pt,a4paper]{extarticle}
\usepackage[utf8]{inputenc}
\usepackage[left=3cm,right=1.5cm,
top=2cm,bottom=2cm,bindingoffset=0cm]{geometry}
\usepackage[english,russian]{babel}
\usepackage[pdftex]{graphicx}
\usepackage{amsfonts}
\usepackage{amsmath}
\usepackage{verbatim}

\pagenumbering{gobble}
\begin{document}
	\begin{center}
			БЕЛАРУСКІ ДЗЯРЖАЎНЫ ЎНІВЕРСІТЭТ \\
			Факультэт прыкладной матэматыкі і інфарматыкі \\
			Кафедра дыскрэтнай матэматыкі і алгарытмікі \\
	\end{center}
	\vspace{30pt}
	\begin{center}
		\textbf{
			ЗАДАННЕ НА КУРСАВЫ ПРАЕКТ \\
		}
	\end{center}
	\vspace{10pt}
	Студэнту \underline{\hspace{5pt}\textit{4}\hspace{5pt}} курса \underline{\hspace{120pt}\textit{Богдану Уладзіславу}\hspace{120pt}} \\
	\renewcommand{\labelenumii}{\arabic{enumi}.\arabic{enumii}.}
	\begin{enumerate}
		\item Тэма праекта: \underline{\textit{Рэканструкцыя паверхні па дадзеных з беспілотных лятальных}} \\
		\underline{\textit{апаратаў}}
		\item Тэрмін здачы студэнтам скончанага праекта: \underline{\textit{снежань 2017}}
		\item Зыходныя дадзеныя для навуковага даследавання (праектавання):
		\begin{enumerate}
			\item \textit{Матэрыялы па агульнай задачы рэканструкцыі паверхні}
			\item \textit{Сучасныя даследаванні па SLAM (Simultanous localization and mapping) алгарытмах: артыкулы, публікацыі}
			\item \textit{SVO (\url{https://github.com/uzh-rpg/rpg\_svo}), а таксама іншыя рэалізацыі SLAM-алгарытмаў}
		\end{enumerate}
		\item Змест курсавога праекта:
		\begin{enumerate}
			\item \textit{Агляд найноўшых публікацыяў па рэканструкцыі паверхні па дадзеных з дронаў}
			\item \textit{Параўнанне магчымасцяў анлайн і афлайн алгарытмаў рэканструкцыі}
			\item \textit{Даследаванне асаблівасцяў SLAM-алгарытмаў}
			\item \textit{Параўнанне існуючых SLAM-рэалізацыяў, ацэнка іх працы ў рэальных умовах, мадыфікацыя з эксперыментальнымі мэтамі}
			\item \textit{Даследаванне магчымасці прымянення выходнай мапы SLAM-алгарытма для павышэння эфектыўнасці стандартных сродкаў глабальнай рэкантрукцыі}
		\end{enumerate}
	\end{enumerate}
	\vspace{30pt}
	Кіраўнік курсавога праекта: $\underset{\text{(подпіс, дата)}}{\underline{\hspace{150pt}}} \hspace{5pt}  /\underline{\hspace{5pt}\textit{Жылка А.І.}\hspace{5pt}}/$ \\
	\vspace{15pt} \\
	Заданне прыняў да выканання: $\underset{\text{(подпіс, дата)}}{\underline{\hspace{150pt}}}$
\end{document}
