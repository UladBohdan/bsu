\begin{center}
    \addcontentsline{toc}{section}{Тэарэтычнае абгрунтаванне}
    \section*{Тэарэтычнае абгрунтаванне}    
\end{center}


(Тут трэба пастарацца і наліць як мага больш вады.

Распавесці пра ўсё і паступова. Пра задачу кампутарнага зроку агулам, пра задачу рэканструкцыі, асобныя часткі пра feature extraction i feature matching, можна з прыкладамі python-кода, дзе праз openCV штось робіцца (альбо не робіцца, ніхто кампіліць не будзе, лол). 

Пасля распавесці што я магу рабіць з дадзенымі, атрыманымі на першых двух этапах. Як працуе рэканструкцыя, чаму гэта магчыма рабіць і чаму гэта працуе, што такое Bundle Adjustment. Распавесці пра агульную задачу і пра нашую, спецыфічную.

Вось не ведаю, сюды гэта, альбо ў практычную частка, але распавесці пра фармат дадзеных, якія мы атрымліваем з дрона, чаму гэта магчыма. Як гэтыя дадзеныя ўплываюць на нашую задачу і на вынікі задачы.

Таксама пра кватэрніоны, што гэта і як мы пераўтвараем іх у тое што нам трэба. Толькі спачатку самому разабрацца, што гэта такое і як гэта працуе.)

----------------------------

Задачу рэканструкцыі паверхні (у агульным выпадку - любога аб'екта) можна падзяліць на некалькі этапаў: выцягванне асаблівых кропак (Feature Extraction), пошук кропак на розных выявах, якія адпавядаюць адной рэальнай кропцы ў прасторы (Feature Matching) і рэканструкцыя паверхні па дадзеных, атрыманых  першых двух этапаў. Пад рэканструкцыяй маецца на ўвазе пабудова воблака кропак, якое пасля можна стаць асновай для пабудовы шчыльнай і трыангуляванай 3d-мадэлі.\\

\addcontentsline{toc}{subsection}{Feature Extraction}
\subsection*{Выцягванне асаблівых кропак}
Само па сабе выцягванне асаблівых кропак з выявы (англ. Feature Extraction) з'яўляецца базавай аперацыяй такой галіны інфарматыкі, як кампутарны зрок (Computer Vision). Сутнасць у вылучэнні кропак на выяве і ў пастаноўцы ім у адпаведнасць дэскрыптара, які апісывае гэтую кропку. Выгляд дэскрыптара адрозніваецца ў залежнасці ад алгарытма, які мы прымяняем для выцягвання асаблівых кропак. Пералічым асноўныя алгарытмы і ідэі іх рэалізацыі. 

\addcontentsline{toc}{subsection}{Feature Matching}
\subsection*{Пошук адпаведных асаблівых кропак}
Пасля таго, як для кожнай выявы з набора мы атрымалі мноства асаблівых кропак, перад намі ставіцца задача пошуку кропак, якія адпавядаюць адным і тым жа кропкам у рэальнай прасторы. Праз тое, што кожная кропка апісываецца сваім дэскрыптарам, мы шукаем кропкі з найменшай адлегласцю (адлегласць вызначаецца адрозным чынам у залежнасці ад таго, які тып дэскрыптара мы выкарыстоўваем) і аб'яўляем, што гэтыя пары (тройкі, і г.д.) кропак на выявах адпавядаюць адной кропцы ў прасторы. Такія мноствы кропак называюцца трэкамі.

\addcontentsline{toc}{subsection}{Reconstruction}
\subsection*{Агульная задача рэканструкцыя паверхні}
Пад рэканструкцыяй паверхні мы маем на ўвазе пабудовы воблака кропак - то бок мноства трохмерных кропак, якія суадносяцца паміж сабой як рэальныя кропцы ў прасторы, у выніку чаго воблака кропак дае нам уяўленне пра ўзаемнае размяшчэнне аб'ектаў на сцэне. Наступным этапам рэканструявання паверхні можа стаць "ушчыльненне" воблака і атрыманне сапраўднай трохмернай мадэлі, што, аднак, выходзіць за межы нашага праэкта: мы спынімся падрабязней на пабудова разрэджанага воблака кропак. \par
У залежнасці ад набора дадзеных, якімі мы распараджаемся, задача рэканструкцыі можа адрознівацца. Пачнем з самай агульнай задачы. \par
Пачнем з некалькіх азначэнняў. \par
Камера - \\
Унутраныя параметры камеры - \\
Вонкавыя параметры камеры - \\
Няхай мы маем $n$ выяваў і $n$ адпаведных ім камер. Кожная камера задаецца наступнымі параметрамі: матрыца павароту (rotation matrix) $R$ памера $3 \times 3$, якая мае наступны выгляд: ололо, translation vector $3 \times 1$, focal length and distortion parameters. \par
Пераўтварэнне з міравых каардынатаў, у каардынаты камеры:
$$ Q = RP + t, $$
Perspective division:
\[ q = - \left( \begin{array}{ccc}
Q_x / Q_z \\
Q_y / Q_z \end{array} \right) 
\]
Conversion to pixel coordinates:
$$ p = f \cdot (1 + k_1 \cdot ||q||^2 + k_2 \cdot ||q||^4) \cdot q $$

\subsubsection*{Заданне матрыцы павароту з дапамогай кватэрніонаў}
Альтэрнатывай апісанай вышэй матрыцы паварота памера $3 \times 3$ можа быць апісанне паварота з дапамогай кватэрніонаў.\par
Што такое кватэрніон.\\
Чаму ён зручнейшы і чаму яго ўсе юзаюць.\\
Як выглядаюць формулы пры яго выкарыстанні, як адбываецца пераўтварэнне паміж гэтымі сістэмамі задання. \\

\subsubsection*{Алгарытмы рэканструкцыі паверхні}

\subsection*{Спецыфіка задачы рэканструкцыі па дадзеных з беспілотных апаратаў}
Той факт, што рэканструкцыя адбываецца не на выпадковым наборы дадзеных, пра які няма ніякай дадатковай інфармацыі, дае нам прастору для аптымізацыі працэса рэканструкцыі, скарачэння часу і павелічэння якасці пабудаванай мадэлі. Пералічым асаблівасці нашай задачы ў пераўнанні з агульнай задачай:
\begin{itemize}
    \item Усе здымкі зробленыя адной фізічнай камерай, такім чынам унутраныя параметры ўсіх камераў застаюцца нязменнымі.
    \item Больш за тое, унутраныя камеры параметры застаюцца нязменнымі не толькі ў межах аднаго датасэта, але і для ўсіх здымкаў зробленым адным БЛА з фіксаванай на ім камерай. Гэты факт не уплывае на алгарытм рэканструкцыі, але мае ўплыў на агульны працэс карыстання дронам на практыцы. (муць нейкая, выдалю нахер напэўна)
    \item Апроч камеры ў нашым распараджэнні, у залежнасці ад канфігурацыі БЛА, могуць мецца і іншыя датчыкі: акселерометр, гіраскоп, GPS-датчык і інш. Дадзеныя сабраныя імі, могуць выкарыстоўвацца як для вызначэння вонкавых параметраў камеры, так і для папярэдняга размяшчэння кропак у прасторы.
\end{itemize}

\subsubsection*{Змены ў набор дадзеных і алгарытмы з улікам вышэйсказанага}

\newpage