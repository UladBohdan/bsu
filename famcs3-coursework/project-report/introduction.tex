\begin{center}
    \addcontentsline{toc}{section}{Уводзіны}
    \section*{Уводзіны}    
\end{center}

З шырокім распаўсюдам беспілотных лятальных апаратаў і камераў бытавога прызначэння вырастае патрэбнасць у хуткай і надзейнай апрацоўцы атрыманых дадзеных. Найбольш натуральным і наглядным падыходам у арганізацыі дадзеных (здымкаў, а таксама дадзеных, атрыманых з дадатковых прыладаў, усталяваных на борце лятальнага апарата, такіх як GPS-датчык) з'яўляецца пабудова трохмернай мадэлі мясцовасці.\par
Задача мае вялікае прыкладное значэнне ў разнастайных сферах дзейнасці: картаграфія, сельская гаспадарка, архітэктура, горнадабыўчая дзейнасць, экалогія. Задача можа ўзнікнуць пры ацэнцы пашкоджанняў мясцовасці пасля прыродных катастрофаў, пры маніторынгу будоўлі аб'ектаў, адсочванні стану навакольнага асяроддзя.\par
Перад намі ставіцца задача хутка і якасна правесці рэканструкцыю паверхні, візуалізаваць мясцовасць для наступнага аналізу прадстаўнікам любых сфераў дзейнасці.\par
У гэтым курсавым праэкце будзе разгледжаная як агульная задача рэканструкцыі паверхні па наборы здымкаў, зробленых звычайнай бытавой камерай, так і спосабы аптымізацыі і паскарэння працэса рэканструкцыі, улічваючы спецыфіку нашай задачы.\par

\newpage