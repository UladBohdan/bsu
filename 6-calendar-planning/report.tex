\documentclass{article}

\usepackage{geometry}
 \geometry{
 a4paper,
 total={170mm,257mm},
 left=20mm,
 top=20mm,
 }

\usepackage[utf8]{inputenc}
\usepackage[russian]{babel}

\usepackage{amsfonts}
\usepackage{amsmath}

\usepackage[none]{hyphenat}

\usepackage{color}

\setlength{\parindent}{0pt}
\setlength{\parskip}{1em}

\begin{document}

{\large

Богдан Уладзіслаў

ФПМІ, 3 курс, 3 група

\vspace{5mm}

\textbf{Справаздача па артыкуле}

\vspace{5mm}

Peter Brucker, Rong Qu, Edmund Burke \\
Personnel scheduling: Models and complexity \\
\textit{Кадравае планаванне: мадэлі і складанасць}
}

\vspace{5mm}

Тэмай майго даклада быў разбор і агляд артыкула ``Кадравае планаванне: мадэлі і складанасць'', які ў 2010 годзе
апублікавалі Peter Brucker, Rong Qu, Edmund Burke. Даследаванні ў галіне кадравага планавання праводзіліся на працягу
дзясяткаў гадоў, але, неглядзячы на гэта, узровень даследавальнасці тэмы застаецца на дастаткова нізкім узроўні. Апроч гэтага,
галіна кадравага планавання дагэтуль разглядалася разасоблена, прапаноўваліся розныя мадэлі для канкрэтных задачаў.
Аўтары артыкула робяць спробу ўніфікацыі тэорыі кадравага планавання і распрацоўкі агульнай мадэлі, здольнай апісываць калі не
ўсе магчымыя, то вялікую колькасць задачаў.

Публікацыя можа быць умоўна падзеленая на некалькі частак.

У першай частцы, аўтары даследуюць і апісываюць цяперашні стан (кажучы цяперашні, я маю на ўвазе час выхаду публікацыі)
галіны кадравага планавання, выяўляюць асноўныя недахопы і праблемы ў распрацоўкі тэорыі (асноўнай праблемай вылучаецца
адсутнасць у літаратуры матэматычнай мадэлі для агульнай задачы кадравага планавання). Таксама разглядаецца некалькі тыпаў задач,
якія дагэтуль у тэорыі распрацоўваліся асобна, прыводзіцца іх класіфікацыя:
\begin{itemize}
  \item \textit{permanence centred planning} - задачы, у якіх патрэбнасць у персанале / кадравым складзе
    загадзя вызначаная і зафіксаваная,
  \item \textit{fluctuation centred planning} - задачы, заснаваныя на неўстойлівым попыце (прыклады: задачы вакол
  гандлёвых складоў, дыстрыбуцыйных цэнтраў, рэстаранаў хуткага харчавання),
  \item \textit{mobility centred planning} - задачы звязаныя з транспартам,
  \item \textit{project centred planning} - задачы, якія паўстаюць у кампаніях, праца ў якіх размеркаваная паміж
  ``праэктамі''; натуральны прыклад - IT-кампаніі.
\end{itemize}

Далей у працы прапаноўваецца матэматычная мадэль, з дапамогай якой могуць быць апісаныя ўсе вышэйзгаданыя задачы.
Апісанне мадэлі дастаткова грувасткае, з вялікай колькасцю параметраў, таму прыводзіць яго тут я не буду: мадэль
вельмі добра і акуратна апісаная ў публікацыі. Прыводзяцца прыклады канкрэтных задачаў і іх апісанняў пры дапамозе
апісанай мадэлі: разглядаецца задача тыпу \textit{project centred planning}, дастаткова вядомая задача арганізацыі
раскладу дзяжурстваў медсёстраў (\textit{Nurse rostering problem}), задача з абмежаваннем на змены заданняў і задача
шматдзённага кадравага планавання.

Апошняя частка публікацыі - ацэнка складанасці алгарытмаў рашэння задачаў кадравага планавання. У новых тэрмінах і
пры дапамозе ўведзенай матэматычнай мадэлі фармулююцца і даказываюцца леммы і тэарэмы, якія паказываюць палінаміальную
вырашальнасць некаторых задачаў (у такім выпадку, звычайна, прымяняюцца алгарытмы пошука максімальнай плыні мінімальнага
кошту ў графе), альбо даказываецца NP-паўната / NP-складанасць задачаў.

Аўтары прывялі добрую класіфікацыю задачаў кадравага планавання, з якімі ужо дастаткова шчыльна працавалі даследчыкі,
прапанавалі матэматычную мадэль апісання вялікага класа задачаў кадравага планавання, прывялі фармуліроўкі некаторых
распаўсюджаных задачаў з улікам новай тэрміналогіі, сфармулявалі ўмовы палінаміальнай вырашальнасці / NP-паўнаты
некаторых добра вядомых задачаў. Артыкул цытаваўся ў сотнях наступных публікацыях.

\thispagestyle{empty}

\end{document}
