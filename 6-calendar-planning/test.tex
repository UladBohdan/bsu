\documentclass{article}

\usepackage{geometry}
 \geometry{
 a4paper,
 total={170mm,257mm},
 left=20mm,
 top=20mm,
 }

\usepackage[utf8]{inputenc}
\usepackage[russian]{babel}

\usepackage{amsfonts}
\usepackage{amsmath}

\usepackage[none]{hyphenat}

\usepackage{color}

\setlength{\parindent}{0pt}
\setlength{\parskip}{1em}

\begin{document}

{\large

Богдан Уладзіслаў

ФПМІ, 3 курс, 3 група

\vspace{5mm}

\textbf{Кантрольная работа}

}

\vspace{10mm}

\subsection*{Уваходныя параметры}

Богдан Уладзіслаў Уладзіміравіч, нарадзіўся 10 сакавіка 1997 г. $\Rightarrow z_1 = 1, z_2 = 10, z_3 = 3, z_4 = 6$.

\subsection*{Рашэнні задачаў}

Коды праграмаў, якія выкарыстоўваліся для рашэння задачаў, знаходзяцца ў асобных файлах, далучаных да ліста.

\subsection*{$A_5.$}
Пераборам варыянтаў рашаем задачу каміваяжора, у якой кожны горад можа быць наведаны толькі аднойчы.
Мая рэалізацыя рашае задачу не поўным пераборам варыянтам, а з адсячэннямі: калі мы спрабуем пабудаваць
маршрут, які ўжо на нейкім этапе становіцца даўжэйшым за знойдзены раней, то гэтую галіну вылічэнняў мы адсякаем.
Для маіх уваходных дадзеных атрымліваем наступны адказ:
\begin{verbatim}
  TSP route found!
  Min cost is:  9
  The route is:  1 2 3 5 4 1
\end{verbatim}
Код праграмы ў \textit{a5\_tsp.cpp}.

\subsection*{$A_6.$}
Знаходзім найкарацейшыя шляхі з вяршыні 1 да ўсіх астатніх алгарытмам Дэйкстры.
Для маіх уваходных дадзеных атрымліваем наступны вывад:
\begin{verbatim}
  1, distance:   0.  How to get: 1
  2, distance:   1.  How to get: 1 2
  3, distance:   3.  How to get: 1 2 6 3
  4, distance:   2.  How to get: 1 4
  5, distance:   5.  How to get: 1 5
  6, distance:   2.  How to get: 1 2 6
  7, distance:   4.  How to get: 1 4 7
  8, distance:   3.  How to get: 1 8
  9, distance:   2.  How to get: 1 9
\end{verbatim}
Код праграмы ў \textit{a6\_dijkstra.cpp}.

\subsection*{$B_{3}.$}
Маем 7 прадметаў з памерамі $p_j = 1, 10, 3, 6, 3, 2, 5$, маем кантэйнеры памера $d = 13$.
Дзейнічаем па наступным алгарытме: дадаем прадмет у найбольш загружаны кантэйнер. Тады прадметы
з памерамі $1, 10, 2$ пойдуць у першы кантэйнер, $3, 6, 3$ - у другі, $5$ - у трэці.
Заўважым, што калі папярэдне адсартаваць памеры на неўзрастанні, то нам таксама спатрэбіцца мінімум
3 кантэйнера (хаця размеркаванне прадметаў паміж кантэйнерамі і зменіцца).

\subsection*{$B_{14}.$}
Маем 4 віда прадуктаў з коштамі роўнымі, адпаведна, $5, 20, 9, 24$. Няхай $c$ - вектар коштаў.
$c = [5, 20, 9, 24]^T$. Абазначым праз вектар $x$ - колькасць адзінак кожнага прадукта, $x_i \in \mathbb{Z}$.
Тады астатнія абмежаванні будуць мець выгляд $A^Tx \geq b$, дзе:
\[
A =
 \begin{pmatrix}
  400 & 2 & 2 & 3 \\
  20 & 2 & 4 & 2 \\
  90 & 4 & 1 & 0 \\
  150 & 4 & 5 & 0
 \end{pmatrix},
b =
 \begin{pmatrix}
  1500 \\
  40 \\
  18 \\
  26
 \end{pmatrix}
\]
Пабудавалі матэматычную мадэль задачы цэлалікавага лінейнага праграмавання.

\subsection*{$B_{15}.$}
Значэнні вектара функцый $F = [F_1, F_2, F_3]$ для $x \in {1, 2, 3, 4, 5, 6, 7}$ для маіх значэнняў:
$F(1) = [1, 10, 3], F(2) = [2, 1, 6], F(3) = [3, 10, 3], F(4) = [2, 2, 5], F(5) = [3, 1, 6], F(6) = [1, 4, 7], F(7) = [3, 3, 3]$.

У мноства Парэта ўваходзяць: $F(2), F(5), F(6)$.

У мноства Парэта не ўваходзяць (у квадратных дужках - вектары, якія ``лепш'' за іх):

$F(1) [F(6)], F(4) [F(2)], F(3) [F(2)], F(7) [F(4)]$.

\subsection*{$B_{16}.$}
$a_{z_1, 11} = a_{1, 11} = 1$, таму ў першым радку матрыцы сумежнасці вяршыняў маем пяць ``1''-ак, то бок
вяршыня 1 сумежная з пяццю іншымі, адсюль вынікае, што ў графе няма эйлеравага цыкла праз крытэр
эйлеравасці неарыентаванага графа.

\end{document}
