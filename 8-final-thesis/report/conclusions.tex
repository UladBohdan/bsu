\begin{center}
    \addcontentsline{toc}{section}{ВЫСНОВЫ}
    \section*{ВЫСНОВЫ}
\end{center}

\vspace{5mm}

У дадзенай справаздачы былі апісаныя асноўныя паняцці і канцэпцыі,
датычныя да задачы трохмернай рэканструкцыі паверхні па дадзеных з БПЛА,
сфармуляваная канцэпцыя SLAM, апісаныя некаторыя
найбольш вядомыя і паспяхова рэалізаваныя SLAM-сістэмы.

Негледзячы на тое, што SLAM, рэалізаваны для працы ў рэальным часе, і які часцей за ўсё выкарыстоўваецца
для навігацыі ў прасторы, кепска спраўляецца з задачай пабудовы шчыльнай трохмернай
мадэлі, вынікі адпрацоўкі SLAM могуць быць вельмі карыснымі ў якасці пачатковых
дадзеных для запуску традыцыйных алгарытмаў рэканструкцыі.

У якасці практычнай рэалізацыі было расправацаванае праграмнае забеспячэнне, якое здзяйсняе
поўны цыкл рэканструкцыі па наборы здымкаў з мноствам дадатковага функцыяналу, такога як:
падтрымка праектаў і захаванне іх у файлавую сістэму, візуалізацыя трохмерных мадэляў
і захаванне мноства мадэляў у межах аднаго праекта, выманне і апісанне ключавых кропак,
пошук адпаведнасцяў паміж асаблівымі кропкамі, правядзенне
непасрэдна працэсу рэканструкцыі. Другой практычнай задачай з'яўлялася распрацоўка
сістэмы, якая б інтэгравала ў сябе SLAM-алгарытм для падліку папярэдніх значэнняў некаторых
параметраў, такіх як пазіцыі камераў у прасторы,
якія пасля бы выкарыстоўваліся ўсё тым жа праграмным забеспячэннем
для правядзення рэканструкцыі з большай хуткасцю і якасцю. Была распрацаваная архітэктура
такой сістэмы і рэалізаваная большасць яе кампанентаў, аднак не была дасягнутая
суцэльнасць сістэмы, пры якой можна было б казаць пра паспяховую яе рэалізацыю і дасягненне канчатковай мэты.
Тым не менш, ёсць перакананасць, што SLAM-алгарытмы сапраўды могуць заняць моцныя пазіцыі ў задачах
пабудовы шчыльнай трохмернай рэканструкцыі, нават калі частка абавязкаў па ўшчыльненні мадэлі
будзе ляжаць на алгарытмах, няздольных працаваць у рэальным часе.

\newpage
