\begin{center}
    \renewcommand{\nextTitle}{ГЛАВА 1. АКТУАЛЬНАСЦЬ ЗАДАЧЫ РЭКАНСТРУКЦЫІ ПАВЕРХНІ}
    \addcontentsline{toc}{section}{\nextTitle}
    \section*{\nextTitle}
\end{center}

\vspace{5mm}

\renewcommand{\cursection}{1}
\setcounter{figure}{0}

\renewcommand{\nextTitle}{1.1 Агульныя звесткі}
\addcontentsline{toc}{subsection}{\nextTitle}
\subsection*{\nextTitle}

У агульным выпадку задача рэканструкцыі паверхні фармулюецца наступным чынам:
неабходна рэканструяваць трохмерны аб'ект па мностве зробленых з розных ракурсаў
здымкаў. Задача фармулюецца дастаткова натуральна чынам, і калі чалавеку дастаткова кінуць
позірк на аб'ект, каб уявіць ягоную трохмерную структуру, алгарытмічна задача
ўсё яшчэ застаецца не да канца вырашанай, звычайна патрабуе вялікіх вылічальных магутнасцяў
і не працуе ўніверсальна добра для любых асяроддзяў і любых умоваў здымак, такіх як,
напрыклад, адрозныя па асвятленні сцэны.
Пад трохмерным аб'ектам у залежнасці ад кантэксту могуць мецца на ўвазе адрозныя рэчы:
калі ў некаторых сітуацыях раздрэджанае воблака кропак будзе лічыцца добрым прыкладам
рэканструяванай структуры, то ў іншых пастаўленая задача можа запатрабаваць пабудову
шчыльнай і гладкай мадэлі з нанесенымі тэкстурамі і колерамі. Падбор алгарытмаў і
ацэнка вылічальных магутнасцяў здзяйсняецца ў залежнасці ад пастаўленых патрабаванняў.

Задача рэканструкцыі можа таксама фармулявацца для іншых тыпаў уваходных дадзеных:
апроч той ці іншай камеры ўваходнымі дадзенымі для задачы могуць быць
дадзеныя з іншых датчыкаў, такіх як акселерометр, гіраскоп ці GPS-датчык;
замест манакулярнай камеры можа прымяняцца RGB-D камера (вяртае дадатковы слой глыбіняў)
стэрэа-камера (уяўляе сабой дзве RGB камеры на фіксаванай паміж сабой адлегласці,
якія ў пэўным сэнсе імітуюць бінакулярны чалавечы зрок), альбо, напрыклад, лазерная камера.

Варта дадаць, што даследаванні ў гэтай галіне камп'ютарнага зроку развіваюцца
таксама праз удасканаленне апаратнага забеспячэння: падыходы, якія некалькі год
таму былі практычна нерэалізуемымі і былі магчымыя толькі ў тэорыі,
з развіццём тэхналогіяў атрымліваюць новае жыццё.

Разам з тым, мноства праблемаў застаюцца нявырашанымі. Поспех усёй галіны даследаванняў
залежыць ад таго, наколькі адначасова добра будуць удасканальвацца вылічальныя магутнасці
камп'ютарных сістэм, алгарытмы, а таксама сродкі захопу дадзеных - усё яшчэ мноства праблем
узнікае акурат праз недасканаласць, недакладнасць альбо нестабільнасць камер і іншых датчыкаў.

\renewcommand{\nextTitle}{1.2 Сферы прымянення}
\addcontentsline{toc}{subsection}{\nextTitle}
\subsection*{\nextTitle}

Цікавасць задачы рэканструкцыі таксама ў запатрабаванні атрымання рашэння ў абсалютна розных сферах
жыцця, кожная з якіх дыктуе свае асаблівасці і прымушае развіваць даследаванні ў тым ці іншым кірунку.

БПЛА выкарыстоўваюцца надзвычайнымі службамі для ацэнкі наступстваў прыродных катастроф,
у сельскай гаспадарцы, дарожнымі службамі для маніторынгу і аналізу трафіка.
Патрабаванні да хуткасці працы алгарытмаў рэканструкцыі
натуральныя - хуткасць працы ў некаторых галінах жыцця крытычная і разбор вялікіх аб'ёмаў
неапрацаваных дадзеных можа стацца непераадольна вялікай працай для чалавека. Патрабаванне да алгарытмаў
рэканструкцыі працаваць у рэальным часе ў большасці выпадкаў з'яўляецца пры навігацыі і
аўтаномным руху, у такім выпадку шчыльная рэканструкцыя можа быць залішняй і
разрэджаная мадэль у выглядзе воблака кропак цалкам задаволіць патрабаванні. У такіх выпадках
мы часта кажам пра пабудову мапы - аб'екта, які ўяўляе сабой своеасаблівую мадэль навакольнага
свету, з пазначанымі на ім кропкамі, якія прадстаўляюць для нас цікаўнасць. Мапы часта
выкарыстоўваюцца для лакалізацыі і навігацыі і могуць быць схаваныя ад вачэй назіральніка.
Падрабязней да тэмы пабудовы мапаў мы вернемся падчас абмеркавання SLAM-алгарытмаў.

\renewcommand{\nextTitle}{1.3 Асаблівасці задачы пры выкарыстанні БПЛА}
\addcontentsline{toc}{subsection}{\nextTitle}
\subsection*{\nextTitle}

Той факт, што рэканструкцыя адбываецца не на выпадковым наборы дадзеных, пра які адсутнічае дадатковая інфармацыя,
дае нам прастору для аптымізацыі працэса рэканструкцыі: скарачэнне часу і паляпшэнне якасці пабудаванай мадэлі.
Пералічым асаблівасці задачы рэканструкцыі па дадзеных з БПЛА ў параўнанні з агульнай задачай:
\begin{itemize}
    \item Усе здымкі зробленыя адной фізічнай камерай, такім чынам унутраныя параметры ўсіх камераў застаюцца нязменнымі.
    Больш за тое, унутраныя параметры камеры застаюцца нязменнымі не толькі ў межах аднаго набора,
    але і для ўсіх здымкаў зробленым адным БПЛА з зафіксаванай на ім камерай.
    Апошні факт не ўносіць зменаў у працэс рэканструкцыі, але спрашчае працэс выкарыстання БПЛА на практыцы
    (адсутнасць патрэбы ў шторазовым калібраванні для высвятлення ўнутраных параметраў камеры).
    \item Апроч камеры ў нашым распараджэнні, у залежнасці ад канфігурацыі БПЛА, могуць мецца і іншыя датчыкі:
    акселерометр, гіраскоп, GPS-датчык і інш.
    Дадзеныя сабраныя імі, могуць выкарыстоўвацца як для вызначэння вонкавых параметраў камеры,
    так і для папярэдняга размяшчэння кропак у прасторы.
\end{itemize}

\subsubsection*{Патэнцыйныя змены ў набор дадзеных і алгарытмы з улікам вышэйсказанага}

Прычыны, па якіх дадатковыя дадзеныя станоўча паўплываюць на якасць і хуткасць рэканструкцыі, абсалютна натуральныя:
памяншэнне колькасці параметраў у глабальнай аптымізацыйнай задачы і больш якасная пачатковая апраксімацыя.
Такім чынам мы маем два падыходы да выкарыстання дадзеных, якімі суправаджаецца набор выяваў:
\begin{itemize}
    \item Аб'яўленне параметраў канстантнымі і непасрэдная іх падстаноўка ва ўраўненні - такім чынам,
    маем яўнае памяншэнне колькасці параметраў што не можа не паўплываць на хуткасць адпрацоўкі алгарытма
    і на выніковае значэнне памылкі праекцыі станоўча.
    Відавочны і недахоп: немагчымасць дакладнага папярэдняга задання параметраў і наступнае аб'яўленне іх канстантнымі
    прывядзе да псавання канчатковых значэнняў трохмерных кропак і, адпаведна, пагаршэння якасці пабудаванай мадэлі.
    Дакладнасць вызначэння параметраў знаходзіцца ў простай залежнасці ад карэктнасці пабудаванай мадэлі.
    На практыцы высвятляецца, што такі падыход вядзе да значнага пагаршэння канчатковых вынікаў
    і кепска прымяняльны на рэальных дадзеных.
    \item Выкарыстанне дадатковых дадзеных для пачатковай апраксімацыі значэнняў параметраў камеры і
    палажэнняў кропак у прасторы. Пры такім падыходзе, мы не рызыкуем стаць ахвярамі кепскай дакладнасці
    альбо памылак у апісанні дадзеных - разам з тым, колькасць ітэрацый у працэсе пучковай аптымізацыі
    можа паменшыцца ў разы. Гэты падыход можа паспяхова быць прыменены на практыцы:
    хуткасць працэса пабудовы проста залежыць ад якасці ўваходных дадзеных -
    яўных жа памылак у выніковай мадэлі (такіх, як у першым пункце) назірацца не будзе,
    бо алгарытм выправіць яўна хібныя ўваходныя дадзеныя.
\end{itemize}

Альтэрнатывай можа стаць змяшаны падыход: напрыклад, аб'явіць фокусную адлегласць пастаяннай
і недаступнай да зменаў (калі мы дастаткова ўпэўненыя ў лічбах, якія прадастаўляе вытворца камера,
альбо атрыманых у выніку каліброўцы), але дазволіць алгарытму удакладняць параметры павароту камеры -
пры дастаткова дакладных пачатковых значэннях яны будуць змененыя нязначна.

Апісаныя вышэй падыходы я выкарыстоўваю ў сваім распрацаваным праграмным забеспячэнні,
падрабязнае апісанне якога прыводзіцца ў апошняй главе.

\newpage
