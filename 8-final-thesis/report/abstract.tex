\begin{titlepage}
    \begin{center}
        \section*{РЭФЕРАТ}
    \end{center}

    \vspace{10mm}

    Дыпломная праца, 41 с., 16 выяў, 5 формул, 18 крыніц

    \vspace{4mm}

    БПЛА, ТРОХМЕРНАЯ РЭКАНСТРУКЦЫЯ, СТЭРЭАБАЧАННЕ, SLAM-АЛГАРЫТМЫ,
    ЭПІПАЛЯРНАЯ ГЕАМЕТРЫЯ, МАНАКУЛЯРНАЯ КАМЕРА, КЛЮЧАВЫЯ КРОПКІ,
    ДЭСКРЫПТАРЫ КЛЮЧАВЫХ КРОПАК, ПОШУК АДПАВЕДНАСЦЯЎ

    \vspace{4mm}

    Аб'ектам даследвання з'яўляюцца алгарытмы трохмернай рэканструкцыі паверхні
    па дадзеных з беспілотных лятальных апаратаў.

    \vspace{4mm}

    Мэта працы –- даследаваць будову алгарытмаў рэканструкцыі паверхні па наборы здымкаў,
    правесці аналіз існуючых SLAM-алгарытмаў, разгледзець магчымасці інтэграцыі двух падыходаў
    да рэканструкцыі паверхні, расправаць адпаведнае праграмнае забеспячэнне.

    \vspace{4mm}

    Метады даследвання: аналіз публікацыяў і вынікаў эксперыментаў,
    даследванне ўнутранай будовы алгарытмаў,
    правядзенне уласных эксперыментаў, распрацоўка праграмнага забеспячэння.

    \vspace{4mm}

    У ходзе працы атрыманыя наступныя вынікі:

    \vspace{4mm}

    \begin{enumerate}
        \item Праведзены аналіз і прыведзеная апісанне сучасных SLAM-алгарытмаў.
        \item Прапанаваныя падыходы, пры якіх SLAM-алгарытмы могуць быць інтэграваныя з
        класічнымі алгарытмамі для рэканструкцыі паверхні па дадзеных з беспілотных лятальных апаратаў.
        \item Распрацаванае праграмнае забеспячэнне для маніпуляцыі з наборамі дадзеных,
        пабудовы трохмерных мадэляў па наборы здымкаў, візуалізацыі. Праграмнае забеспячэнне,
        апроч рэалізацыі традыцыйных падыходаў да рэканструкцыі, інтэграванае з SLAM-сістэмамі,
        адкуль яно вымае дадатковыя дадзеныя і выкарыстоўвае для ўдасканалення сваёй працы.
    \end{enumerate}

    \vspace{4mm}

    Галіны прымянення: камп'ютарны зрок, трохмерная рэканструкцыя, іншыя прыкладанні.
\end{titlepage}

\newpage

\begin{titlepage}
    \begin{center}
        \section*{РЕФЕРАТ}
    \end{center}

    \vspace{10mm}

    Дипломная работа, 40 с., 16 изображений, 5 формул, 18 источников

    \vspace{4mm}

    БПЛА, ТРЁХМЕРНАЯ РЕКОНСТРУКЦИЯ, СТЕРЕОМЕТРИЯ, SLAM-АЛГОРИТМЫ, ЭПИПОЛЯРНАЯ ГЕОМЕТРИЯ,
    МОНОКУЛЯРНАЯ КАМЕРА, КЛЮЧЕВЫЕ ТОЧКИ, ДЕСКРИПТОРЫ КЛЮЧЕВЫХ ТОЧЕК, ПОИСК СООТВЕТСТВИЙ

    \vspace{4mm}

    Объектом исследования являются алгоритмы трёхмерной реконструкции поверхности
    по данным с беспилотных летательных аппаратов.

    \vspace{4mm}

    Цель работы –- исследовать строение алгоритмов реконструкции поверхности по набору снимков,
    провести анализ существующих SLAM-алгоритмов, рассмотреть возможности интеграции двух подходов
    к реконструкции поверхности, разработать соответствующее программное обеспечение.

    \vspace{4mm}

    Методы исследования: анализ публикаций и результатов экспериментов,
    исследование внутреннего строения алгоритмов, проведение собственных экспериментов,
    разработка программного обеспечения.

    \vspace{4mm}

    В ходе работы получены следующие результаты:

    \vspace{4mm}

    \begin{enumerate}
        \item Проведён анализ и приведено описание современных SLAM-алгоритмов.
        \item Предложены подходы, при которых SLAM-алгоритмы могут быть интегрированы с классическими
        алгоритмами для реконструкции поверхности по данным с беспилотных летательных аппаратов.
        \item Разработано программное обеспечение для манипуляции с наборами данных,
        построения трёхмерных моделей па набору снимков, визуализации.
        Программное обеспечение, кроме реализации традиционных подходов к реконструкции,
        интегрировано с SLAM-системами, откуда оно достаёт дополнительные данные и использует для своей работы.
    \end{enumerate}

    \vspace{4mm}

    Область применения: компьютерное зрение, трёхмерная реконструкция, приложения.
\end{titlepage}

\newpage

\begin{titlepage}
    \begin{center}
        \section*{ABSTRACT}
    \end{center}

    \vspace{10mm}

    Graduate work, 41 p., 16 figures, 5 formulas, 18 sources

    \vspace{4mm}

    UAV, THREE-DIMENSIONAL RECONSTRUCTION, STEREOVISION, SLAM ALGORITHMS,
    EPIPOLAR GEOMETRY, MONOCULAR CAMERA, KEYPOINTS, FEATURES, FEATURES’ DESCRIPTORS,
    FEATURE MATCHING

    \vspace{4mm}

    The object of the research are algorithms of three-dimensional reconstruction
    of the surface based on the data from unmanned aerial vehicles.

    \vspace{4mm}

    The purpose –- to research the structure of the algorithms of surface reconstruction
    from a set of images, to analyze existing SLAM algorithms,
    to search for the possibilities to integrate two reconstruction approaches, to develop the software.

    \vspace{4mm}

    Methods of the research are: to analyze publications and results of the experiments,
    to research internal structure of the algorithms, to perform own experiments, to develop the software.

    \vspace{4mm}

    During the research the following results were obtained:

    \vspace{4mm}

    \begin{enumerate}
        \item Analysis of state-of-art SLAM algorithms was performed and the description was provided.
        \item The ways to integrate SLAM algorithms with traditional algorithms
        for surface reconstruction based on the data from UAV were suggested.
        \item The software for data manipulation, three-dimensional models’
        construction and visualization was developed. Furthermore, the software was
        integrated with SLAM systems in order to obtain data to improve its work.
    \end{enumerate}

    \vspace{4mm}

    The scopes are: computer vision, three-dimensional reconstruction, other applications.
\end{titlepage}

\newpage
