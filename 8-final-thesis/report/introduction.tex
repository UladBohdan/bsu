\begin{center}
    \addcontentsline{toc}{section}{УВОДЗІНЫ}
    \section*{УВОДЗІНЫ}
\end{center}

\vspace{5mm}

Задача рэканструкцыі паверхні зямлі па дадзеных з беспілотных лятальных апаратаў
(БПЛА) узнікае ў вялікай колькасці сфераў жыццядзейнасці: ад сельскай гаспадаркі да
ацэнкі наступстваў прыродных катастрофаў. З кожным годам усё больш шырокае развіццё ў свеце
атрымліваюць аўтаномныя аўтамабілі, сродкі сачэння ў месцах вялікага скаплення людзей.
Патрэбнасць у вырашэнні тых ці іншых задач,
якія патрабуюць прымянення беспілотнага лятальнага апарата (дрона) можа ўзнікнуць
у арганізацыяў розных памераў.

Сам па сабе беспілотны лятальны апарат звычайна выступае толькі у якасці пачатковага
``зборшчыка'' інфармацыі, тады як задача апрацоўкі, удасканалення і ўзгаднення інфармацыі
стаіць асобна і патрабуе распрацоўкі эфектыўных, хуткіх і надзейных алгарытмаў.
Згаданыя вышэй задачы штогод атрымліваюць усё большы распаўсюд, у тым ліку ў сувязі
з шырокай даступнасцю БПЛА і з распаўсюдам танных камераў, якімі ўкамплектоўваецца борт БПЛА.

Задача рэканструкцыі цесна звязаная са шматлікімі сумежнымі задачамі:
распазнаванне аб'ектаў, навігацыя ў прасторы, пабудова мапы мясцовасці.
Асаблівую цікаўнасць прадстаўляюць рашэнні, якія выконваюцца ў рэальным часе;
праца ў рэальным часе крытычная для алгарытмаў навігацыі і абыходу перашкодаў
пры запуску на аўтаномных БПЛА, бо аўтаномнасць вымагае хуткага і беспамылковага рэагавання
на змены ў навакольным асяроддзі.

Алгарытмы, якія прымяняюцца пры здзяйсненні працэса рэканструкцыі, могуць быць умоўна падзеленыя на тыя,
што выконваюцца пры наяўнасці поўнага набора дадзеных (``афлайн''-алгарытмы), і тыя, якія выконваюцца ў рэальным
часе і аперуюць з дадзенымі, якія паступаюць у ``анлайн'' рэжыме. Да апошніх, у прыватнасці,
адносяцца алгарытмы, пабудаваныя па канцэпцыі SLAM (англ. \textit{Simultaneous
Localization and Mapping}) - алгарытмы адначасовай лакалізацыі і пабудовы мапы.
SLAM-алгарытмы атрымалі моцны штуршок у развіцці ў апошнія гады ў сувязі з прэзентацыяй
шэрага сістэм, якія моцна пераўзыходзяць усе папярэднія напрацоўкі. У наступных главах,
сярод іншага, падрабязна распавядаецца пра некаторыя сучасныя SLAM-сістэмы.

Мэтамі распрацоўкі дадзенай тэмы ставіліся даследаванне тэарэтычных аспектаў задачы
рэканструкцыі, а таксама практычная рэалізацыя шэрага праграмных модуляў, якія бы
здзяйснялі рэканструкцыю паверхні. Пры гэтым дадзеныя, згенераваныя SLAM
алгарытмамі, прапаноўваецца выкарыстоўваць для аптымізацыі гэтага працэса.

У гэтай працы быў зроблены агляд разнастайных падыходаў да рэканструкцыі паверхні,
прыведзенае параўнанне алгарытмаў, якія працуюць у рэальным часе, з тымі, што працуюць
з папярэдне сабранымі БПЛА дадзенымі, разгледжаная магчымасць аб'яднання
розных падыходаў у адну сістэму. У практычнай частцы прыводзіцца справаздача па распрацаваным
праграмным забеспячэнні для рэканструкцыі паверхні па наборы здымкаў і па дадатковых дадзеных (пры наяўнасці),
а таксама прапаноўваецца і прыводзіцца статус распрацаванасці архітэктуры сістэмы,
якая спалучае ў сабе SLAM-падыходы з ``афлайн'' падыходамі для
здзяйснення эфектыўнай рэканструкцыі паверхні па дадзеных з беспілотных лятальных апаратаў.

\newpage
