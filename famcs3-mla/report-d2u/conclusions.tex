\section*{Высновы}
Можна рашыць дыферэнцыяльнае ўраўненне і атрымаць дакладнае рашэнне:
\begin{equation} \label{eq:d2u-exact}
    u(x) = 1.141 e^{-1.618x} + 0.339 e^{0.618x} - e^{-2x}
\end{equation}
Ва ўраўненні \eqref{eq:d2u-exact} я замяніў дакладныя значэнні каэфіцыэнтаў на набліжэнні ў выглядзе дзесятковых дробаў для спрашчэння падлікаў - на канчатковы вынік падобная аперацыя будзе мець мінімальны ўплыў.\par
Падлічым дакладныя значэнні на сетцы:
{\small
\begin{verbatim}
Exact solution:
y(0.0) = 0.48
y(0.1) = 0.512427856823
y(0.2) = 0.53883854757
y(0.3) = 0.561472157666
y(0.4) = 0.582062960216
y(0.5) = 0.601950592919
y(0.6) = 0.622168463031
y(0.7) = 0.643513918114
y(0.8) = 0.666603837732
y(0.9) = 0.691918587652
y(1.0) = 0.719836701642
\end{verbatim}
}

\newpage

Знойдзем хібнасці абодвух метадаў, то бок для кожнага $x_i$ з сеткі знойдзем розніцу $u_i - y_i = u(x_i) - y(x_i)$, дзе $u_i$ - дакладнае значэнне, $y_i$ - набліжэнне.\par
\vspace{5mm}
Метад Рытца:
{\small
\begin{verbatim}
Ritz:
u(0.0) - y(0.0) = 0.0
u(0.1) - y(0.1) = 0.00259706290159
u(0.2) - y(0.2) = 0.000844520904628
u(0.3) - y(0.3) = 0.00100410377728
u(0.4) - y(0.4) = 0.00133443716583
u(0.5) - y(0.5) = 0.0006432081075
u(0.6) - y(0.6) = -0.000218143860458
u(0.7) - y(0.7) = -0.000159727346558
u(0.8) - y(0.8) = 0.000244107235884
u(0.9) - y(0.9) = -0.000597435135413
u(1.0) - y(1.0) = 0.0
\end{verbatim}
}

Метад сетак $O(h^2)$:
{\small
\begin{verbatim}
Grids:
u(0.0) - y(0.0) = -0.00182697281369
u(0.1) - y(0.1) = -0.010813171041
u(0.2) - y(0.2) = -0.0159333852899
u(0.3) - y(0.3) = -0.0181692000061
u(0.4) - y(0.4) = -0.0182785925757
u(0.5) - y(0.5) = -0.0168430846996
u(0.6) - y(0.6) = -0.0143053390252
u(0.7) - y(0.7) = -0.0109990831515
u(0.8) - y(0.8) = -0.00717287937789
u(0.9) - y(0.9) = -0.00300896413092
u(1.0) - y(1.0) = 0.00136185717777
\end{verbatim}
}

Метад Рытца ў розных вяршынях сеткі дае хібнасць у межах $0.01 - 0.001$, метад сетак атрымаўся менш дакладным, але заяўленая дакладнасць (2-гі парадак) дасягаецца. Кожны з гэтых алгарытмаў патрабуе рэалізацыі дадатковых функцый - падлік інтэграла альбо рэалізацыя метада прагонкі. Кожны з іх можа быць паспяхова прыменены на практыцы.
