\section{Задача Дырыхле для ўраўнення Пуасона}

\subsection*{Пастаноўка задача}
Дадзеная задача Дырыхле для ўраўнення Пуасона на простакутніку $[a,b]\times[c,d]$:
\begin{equation} \label{eq:dirichlet-general}
    \frac{\partial^2(u)}{\partial x^2} + \frac{\partial^2(u)}{\partial y^2} = -f(x,y),
\end{equation}
\[ x \in [a, b], y \in [c, d] \]
\[ u(a, y) = \psi_1(y), \hspace{10pt} u(b, y) = \psi_2(y) \]
\[ u(x, c) = \psi_3(x), \hspace{10pt} u(x, d) = \psi_4(x) \]
Ставіцца мэта знайсці набліжанае рашэнне рознасным ітэрацыйным метадам на сетцы вузлоў з крокамі $h_x = h_y = 0.05$.\par
\vspace{5mm}
У маім выпадку задача мае наступны выгляд:
\begin{equation}
    \frac{\partial^2(u)}{\partial x^2} + \frac{\partial^2(u)}{\partial y^2} = - | sin^3(\pi x y) |
\end{equation}
\[ x \in [-1, 1], y \in [-1, 1] \]
\[ u(a, y) = \psi_1(y) = 1 - y^2, \hspace{10pt} u(b, y) = \psi_2(y) = 1 - y^2 \]
\[ u(x, c) = \psi_3(x) = |sin(\pi x)|, \hspace{10pt} u(x, d) = \psi_4(x) = |sin(\pi x)| \]


\subsection*{Абгрунтаванне метада}

Апраксімуем другія вытворныя ў \eqref{eq:dirichlet-general} з дапамогай рознасных схемаў. Атрымаем:
\begin{equation}
    (y_{\overline{x}})_{x} + (y_{\overline{y}})_{y} = -f
\end{equation}
\[ y_{0j} = \psi_{1j}, \hspace{10pt} y_{N_x,j} = \psi_{2j}, \hspace{10pt} j = \overline{0, N_y} \]
\[ y_{i0} = \psi_{3i}, \hspace{10pt} y_{i,N_y} = \psi_{4i}, \hspace{10pt} i = \overline{0, N_x} \]

Распішам у індэкснай форме:
\begin{equation} \label{eq:dirichlet-index}
    \frac{y_{i+1,j} - 2y_{ij} + y_{i-1,j}}{h_{x}^2} + \frac{y_{i,j+1} - 2y_{ij} + y_{i,j-1}}{h_{y}^2} = -f_{ij}, \hspace{5pt} i = \overline{1, N_x-1}, j = \overline{1, N_y-1}
\end{equation}
Знойдзем хібнасць такой рознаснай схемы - раскладзем $\psi = (u_{\overline{x}})_{x} + (u_{\overline{y}})_{y} + f$:
\begin{equation}
    \psi = \frac{\partial^2u(x_i, y_j)}{\partial x^2} + \frac{h_x^2}{12} \frac{\partial^4u(x_i, y_j)}{\partial x^4} + \mathcal{O}(h_x^4) + \\
    \frac{\partial^2u(x_i, y_j)}{\partial y^2} + \frac{h_y^2}{12} \frac{\partial^4u(x_i, y_j)}{\partial y^4} + \mathcal{O}(h_y^4) + f_{ij} = \mathcal{O}(h_{x}^2 + h_{y}^2)
\end{equation}

Выразім з формулы \eqref{eq:dirichlet-index} $y_{ij}$:
\begin{equation} \label{eq:yij}
    y_{ij} = (\frac{2}{h_{x}^2} + \frac{2}{h_{y}^2})^{-1} ( f_{ij} + \frac{y_{i+1,j} + y_{i-1,j}}{h_{x}^2} + \frac{y_{i,j+1} + y_{i,j-1}}{h_{y}^2} ), \hspace{5pt} i = \overline{1, N_x-1}, j = \overline{1, N_y-1}
\end{equation}

Формула \eqref{eq:yij} можа быць падлічаная любым з ітэратыўных метадаў рашэння нелінейных ураўненняў - па ўмовах маёй задачы, будзем прымяняць метад Зэйдэля:
\begin{equation} \label{eq:yij-iterative}
    y_{ij}^{k+1} = (\frac{2}{h_{x}^2} + \frac{2}{h_{y}^2})^{-1} ( f_{ij} + \frac{y_{i+1,j}^k + y_{i-1,j}^{k+1}}{h_{x}^2} + \frac{y_{i,j+1}^k + y_{i,j-1}^{k+1}}{h_{y}^2} ), \hspace{5pt} i = \overline{1, N_x-1}, j = \overline{1, N_y-1}
\end{equation}

Такім чынам, мы будзем рухацца па сетцы, з кута $(a, c)$ у кут $(b, d)$, першасна ўздоўж восі $Ox$, другасна - уздоўж $Oy$, паслядоўна для кожнага вузла выконваючы \eqref{eq:yij-iterative}. Спыняемся, калі $\max |y_{ij}^{k+1} - y_{ij}^k | \leq \epsilon, i = \overline{1, N_x-1}, j = \overline{1, N_y-1}$.
Каб забяспечыць $\mathcal{O}(h_{x}^2 + h_{y}^2)$ бярэм $\epsilon = \mathcal{O}(h^3)$ (каб дакладна забяспечыць $\epsilon = \mathcal{O}(h^2)$), $h = \min(h_x, h_y)$.


\subsection*{Рэалізацыя}
{\small
\begin{verbatim}
# Defining the Dirichlet problem for Poisson equation.
f = lambda x, y: abs(math.sin(math.pi * x * y) ** 3)
psi_left = lambda y: 1 - y * y
psi_right = lambda y: 1 - y * y
psi_bottom = lambda x: abs(math.sin(math.pi * x))
psi_top = lambda x: abs(math.sin(math.pi * x))
a = -1.
b = 1.
c = -1.
d = 1.
# Algorithm parameters.
hx = 0.2
hy = 0.2
nx = int((b - a) / hx + 1)
ny = int((d - c) / hy + 1)
EPS = 1e-4
MAX_ZEIDEL_ITERATIONS = 1000

def dirichlet():
    grid = [[0. for x in range(nx)] for y in range(ny)]

    # Filling grid with border values.
    for i in range(0, nx):
        x = a + i * hx
        grid[0][i] = psi_bottom(x)
        grid[ny-1][i] = psi_top(x)
    for j in range(0, ny):
        y = c + j * hy
        grid[j][0] = psi_left(y)
        grid[j][nx-1] = psi_right(y)
    # Filling inner cells with f(i,j) - approximating.
    for i in range(1, nx-1):
        x = a + i * hx
        for j in range(1, ny-1):
            y = c + j * hy
            grid[i][j] = f(x, y)

    # Updating all inner cells.
    iter = MAX_ZEIDEL_ITERATIONS
    while iter > 0:
        mx_diff = 0
        for i in range(1, nx-1):
            x = a + i * hx
            for j in range(1, ny-1):
                y = c + j * hy
                oldValue = grid[i][j]
                grid[i][j] = 1. / (2/(hx*hx) + 2/(hy*hy)) * (f(x, y) + (grid[i-1][j] +
                        grid[i+1][j])/hx + (grid[i][j+1] + grid[i][j-1])/hy)
                mx_diff = max(mx_diff, abs(oldValue - grid[i][j]))
        if mx_diff < EPS:
            break
        iter -= 1
    if iter == 0:
        print "Zeidel: no convergence"
        return
    else:
        print "Number of iterations:", MAX_ZEIDEL_ITERATIONS - iter

    for i in range(nx-1, -1, -1):
        for j in range(0, ny):
            value = '{0:.7f} '.format(grid[i][j])
            sys.stdout.write(value)
        sys.stdout.write('\n')
    return
\end{verbatim}
}


\subsection*{Вынік}

Сеткі, прадстаўленыя ніжэй, арыентаваныя так, што ў левым ніжнім куце знаходзіцца кропка $(a, c)$, у правым верхнім - $(b, d)$.\par
\vspace{5mm}

Для сеткі з крокам $h_x = h_y = 0.2$:
{\footnotesize
\begin{verbatim}
0.0000000 0.5877853 0.9510565 0.9510565 0.5877853 0.0000000 0.5877853 0.9510565 0.9510565 0.5877853 0.0000000
0.3600000 0.0604217 0.0641429 0.0588235 0.0337566 0.0034073 0.0337526 0.0588207 0.0641423 0.0604205 0.3600000
0.6400000 0.0483340 0.0137298 0.0070617 0.0026013 0.0004726 0.0026170 0.0070696 0.0137290 0.0483243 0.6400000
0.8400000 0.0535519 0.0068643 0.0018596 0.0003962 0.0001096 0.0004614 0.0019084 0.0068635 0.0535175 0.8400000
0.9600000 0.0548107 0.0037883 0.0004563 0.0000689 0.0000338 0.0001263 0.0005278 0.0037978 0.0547691 0.9600000
1.0000000 0.0556680 0.0031904 0.0002132 0.0000197 0.0000096 0.0000403 0.0002532 0.0032100 0.0556459 1.0000000
0.9600000 0.0547803 0.0037716 0.0004527 0.0000646 0.0000124 0.0000673 0.0004609 0.0037818 0.0547733 0.9600000
0.8400000 0.0535216 0.0068545 0.0018577 0.0003865 0.0000694 0.0003895 0.0018574 0.0068544 0.0535201 0.8400000
0.6400000 0.0483285 0.0137286 0.0070606 0.0025909 0.0004533 0.0026078 0.0070697 0.0137282 0.0483246 0.6400000
0.3600000 0.0604334 0.0641468 0.0588249 0.0337636 0.0034294 0.0337945 0.0588554 0.0641515 0.0604213 0.3600000
0.0000000 0.5877853 0.9510565 0.9510565 0.5877853 0.0000000 0.5877853 0.9510565 0.9510565 0.5877853 0.0000000
\end{verbatim}
}

Для сеткі з крокам $h_x = h_y = 0.1$:
{\tiny
\begin{verbatim}
0.0000 0.3090 0.5878 0.8090 0.9511 1.0000 0.9511 0.8090 0.5878 0.3090 0.0000 0.3090 0.5878 0.8090 0.9511 1.0000 0.9511 0.8090 0.5878 0.3090 0.0000
0.1900 0.0136 0.0168 0.0233 0.0276 0.0288 0.0270 0.0224 0.0159 0.0082 0.0004 0.0082 0.0159 0.0224 0.0270 0.0288 0.0276 0.0233 0.0168 0.0136 0.1900
0.3600 0.0109 0.0027 0.0032 0.0034 0.0031 0.0023 0.0015 0.0007 0.0003 0.0000 0.0003 0.0007 0.0015 0.0023 0.0031 0.0034 0.0032 0.0027 0.0109 0.3600
0.5100 0.0155 0.0030 0.0028 0.0025 0.0020 0.0013 0.0007 0.0002 0.0000 0.0000 0.0000 0.0002 0.0007 0.0013 0.0020 0.0025 0.0028 0.0030 0.0155 0.5100
0.6400 0.0195 0.0032 0.0025 0.0021 0.0015 0.0009 0.0004 0.0001 0.0000 0.0000 0.0000 0.0001 0.0004 0.0009 0.0015 0.0021 0.0025 0.0032 0.0195 0.6400
0.7500 0.0223 0.0029 0.0020 0.0015 0.0010 0.0006 0.0003 0.0001 0.0000 0.0000 0.0000 0.0001 0.0003 0.0006 0.0010 0.0015 0.0020 0.0029 0.0223 0.7500
0.8400 0.0241 0.0023 0.0013 0.0009 0.0006 0.0003 0.0001 0.0000 0.0000 0.0000 0.0000 0.0000 0.0001 0.0003 0.0006 0.0009 0.0013 0.0023 0.0241 0.8400
0.9100 0.0251 0.0015 0.0007 0.0004 0.0003 0.0001 0.0001 0.0000 0.0000 0.0000 0.0000 0.0000 0.0001 0.0001 0.0003 0.0004 0.0007 0.0015 0.0251 0.9100
0.9600 0.0257 0.0010 0.0002 0.0001 0.0001 0.0000 0.0000 0.0000 0.0000 0.0000 0.0000 0.0000 0.0000 0.0000 0.0001 0.0001 0.0002 0.0010 0.0257 0.9600
0.9900 0.0261 0.0007 0.0001 0.0000 0.0000 0.0000 0.0000 0.0000 0.0000 0.0000 0.0000 0.0000 0.0000 0.0000 0.0000 0.0000 0.0001 0.0007 0.0261 0.9900
1.0000 0.0263 0.0007 0.0000 0.0000 0.0000 0.0000 0.0000 0.0000 0.0000 0.0000 0.0000 0.0000 0.0000 0.0000 0.0000 0.0000 0.0000 0.0007 0.0263 1.0000
0.9900 0.0261 0.0007 0.0001 0.0000 0.0000 0.0000 0.0000 0.0000 0.0000 0.0000 0.0000 0.0000 0.0000 0.0000 0.0000 0.0000 0.0001 0.0007 0.0261 0.9900
0.9600 0.0257 0.0010 0.0002 0.0001 0.0001 0.0000 0.0000 0.0000 0.0000 0.0000 0.0000 0.0000 0.0000 0.0000 0.0001 0.0001 0.0002 0.0010 0.0257 0.9600
0.9100 0.0251 0.0015 0.0007 0.0004 0.0003 0.0001 0.0001 0.0000 0.0000 0.0000 0.0000 0.0000 0.0001 0.0001 0.0003 0.0004 0.0007 0.0015 0.0251 0.9100
0.8400 0.0241 0.0023 0.0013 0.0009 0.0006 0.0003 0.0001 0.0000 0.0000 0.0000 0.0000 0.0000 0.0001 0.0003 0.0006 0.0009 0.0013 0.0022 0.0241 0.8400
0.7500 0.0223 0.0029 0.0020 0.0015 0.0010 0.0006 0.0003 0.0001 0.0000 0.0000 0.0000 0.0001 0.0003 0.0006 0.0010 0.0015 0.0020 0.0029 0.0223 0.7500
0.6400 0.0195 0.0032 0.0025 0.0021 0.0015 0.0009 0.0004 0.0001 0.0000 0.0000 0.0000 0.0001 0.0004 0.0009 0.0015 0.0021 0.0025 0.0032 0.0195 0.6400
0.5100 0.0155 0.0030 0.0028 0.0025 0.0020 0.0013 0.0007 0.0002 0.0000 0.0000 0.0000 0.0002 0.0007 0.0013 0.0020 0.0025 0.0028 0.0030 0.0155 0.5100
0.3600 0.0109 0.0027 0.0032 0.0034 0.0031 0.0023 0.0015 0.0007 0.0003 0.0000 0.0003 0.0007 0.0015 0.0023 0.0031 0.0034 0.0032 0.0027 0.0109 0.3600
0.1900 0.0136 0.0168 0.0234 0.0276 0.0289 0.0270 0.0224 0.0159 0.0082 0.0004 0.0082 0.0159 0.0224 0.0270 0.0289 0.0276 0.0234 0.0168 0.0136 0.1900
0.0000 0.3090 0.5878 0.8090 0.9511 1.0000 0.9511 0.8090 0.5878 0.3090 0.0000 0.3090 0.5878 0.8090 0.9511 1.0000 0.9511 0.8090 0.5878 0.3090 0.0000
\end{verbatim}
}

Сетку, пабудаваную з крокамі $h_x = h_y = 0.05$ занадта складана размясціць праз вялікую колькасць вузлоў.\par
