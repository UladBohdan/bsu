\begin{center}
    \addcontentsline{toc}{section}{Высновы}
    \section*{Высновы}    
\end{center}

Намі была апісаная тэорыя ўсіх этапаў пабудовы разрэджанага воблака кропак на аснове набора выяваў, паказаныя адрозненні ў рэалізацыі агульнай задачы рэканструкцыі ад задачы з дадатковымі дадзенымі, прыведзены прыклад фармату файла з апісаннем дадзеных і праведзеная серыя эксперыментаў з мэтай вызначэння, ці дае прырост у хуткасці і якасці мадыфікаваная версія алгарытма.\\

\subsection*{Прымяненне на практыцы}
На практыцы, выкарыстанне мадыфікаванай версіі алгарытма будзе азначаць наступнае: набор уваходных дадзеных змяшчае апроч выяваў файл з дадатковымі дадзенымі, якія мусяць быць папярэдне запісаныя ў яго ў адпаведным фармаце. Такім чынам у працэс апрацоўкі дадзеных пасля палёту дадаецца дадатковы этап, які патрабуе альбо дадатковага праграмнага забеспячэнне альбо чалавечай працы.\\
trololo

\subsection*{Шляхі развіцця ідэі}
Апроч аптымізацыі ўваходных дадзеных, ёсць прастора для аптымізацыі алгарытма. Напрыклад, рэалізацыя ўсяго алгарытма альбо яго крытычных частак на GPU замест CPU.

\newpage