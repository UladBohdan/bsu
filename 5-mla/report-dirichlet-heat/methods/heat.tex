\section{Рознасныя схемы для ўраўнення цеплаправоднасці}

\subsection*{Пастаноўка задачы}
Знайсці набліжанае рашэнне з дапамогай рознаснай схемы з вагамі $\sigma$ на раўнамернай сетцы вузлоў $\overline{\omega} _{h \tau }$ з хібнасцю апраксімацыі не ніжэй за $\mathcal{O}(h^2 + \tau)$ для задачы выгляду:
\begin{equation}
	\begin{cases}
		\frac{\partial u}{\partial t} = \frac{\partial ^2 u}{\partial ^2 x} + f(x,t), \hspace{10pt} 0 \le x,t \le 1,\\
		u(x,0) = u_0(x), \\
		\alpha _0 u(0,t) + \beta _0 \frac{\partial u(0,t)}{\partial x} = \mu _0(t), \\
		\alpha _1 u(1,t) + \beta _1 \frac{\partial u(1,t)}{\partial x} = \mu _1(t). \\
	\end{cases}
\end{equation}

Мая канкрэтная задача выглядае наступным чынам:
\begin{equation}
	\begin{cases}
		\frac{\partial u}{\partial t} = \frac{\partial ^2 u}{\partial ^2 x} + x^2 - 2t, \hspace{10pt} 0 \le x,t \le 1,\\
		u(x,0) = 0, \\
		u(0,t) = 0, \\
		\frac{\partial u(1,t)}{\partial x} = 2t, \\
        \sigma = 0.5,\\
        h = \tau = 0.05.\\
	\end{cases}
\end{equation}

\subsection*{Абгрунтаванне метада}

Для апраксімацыі будзем выкарыстоўваць рознасную схему з шасцікропкавым шаблонам:
\[ y_t = \sigma y_{x \overline{x}} + (1-\sigma) y_{\overline{x}x} + \varphi \]
Умова ўстойлівасці гэтай схемы:  $\sigma \ge 0.5 - \frac{h^2}{4\tau}$. Калі $\sigma$ адпавядае ўмове ўстойлівасці, а $\varphi \equiv f$, рознасная схема апраксімацыі ўраўнення будзе мець неабходны нам парадак апраксімацыі. \\ \\
Разгледзім гранічныя ўмовы, пабудуем для іх рознасную схему і вызначым для гэтай схемы хібнасць апраксімацыі, разгледзеўшы нявязку на дакладным рашэнні. \par
\vspace{5mm}
Першая гранічная ўмова:
\begin{multline}
\alpha _0 u(x_0,t) + \beta _0 u(x_0,t)_x - \mu _0(t) = \alpha _0 u(x_0,t) + \frac{\beta_0}{h}(u(x_1, t) - u(x_0, t)) - \mu _0(t) = \\
= \alpha _0 u(x_0, t) + \frac{\beta _0}{h}(u(x_0, t) + h \frac{\partial u(x_0, t)}{\partial x} + \frac{h^2}{2} \frac{\partial ^2 u(x_0, t)}{\partial ^2 x} + \mathcal{O}(h^3) - u(x_0, t)) -\mu _0(t) = \\
= \alpha _0 u(x_0, t) + \beta _0 \frac{\partial u(x_0, t)}{\partial x} + \frac{\beta _0 h}{2} \frac{\partial ^2 u(x_0, t)}{\partial ^2 x} + \mathcal{O}(h^2) -\mu _0(t) = \frac{\beta _0 h}{2} \frac{\partial ^2 u(x_0, t)}{\partial ^2 x} + \mathcal{O}(h^2)
\end{multline}

Другая гранічная ўмова:
\begin{multline}
\alpha _1 u(x_N,t) + \beta _1 u(x_N,t)_{\overline{x}} - \mu _1(t) = \alpha _1 u(x_N,t) + \frac{\beta_1}{h}(u(x_N, t) - u(x_{N-1}, t)) - \mu _1(t) = \\
= \alpha _1 u(x_N, t) + \frac{\beta _1}{h}(u(x_N, t) - (u(x_N, t) - h \frac{\partial u(x_N, t)}{\partial x} + \frac{h^2}{2} \frac{\partial ^2 u(x_N, t)}{\partial ^2 x} + \mathcal{O}(h^3))) -\mu _1(t) = \\
= \alpha _1 u(x_N, t) + \beta _1 \frac{\partial u(x_N, t)}{\partial x} - \frac{\beta _1 h}{2} \frac{\partial ^2 u(x_N, t)}{\partial ^2 x} + \mathcal{O}(h^2) -\mu _1(t) = -\frac{\beta _1 h}{2} \frac{\partial ^2 u(x_N, t)}{\partial ^2 x} + \mathcal{O}(h^2)
\end{multline}

Атрымліваецца парадак апраксімацыі $\mathcal{O}(h)$. Каб павысіць парадак да неабходнага, трэба ўвесці ў гранічныя ўмовы галоўны член хібнасці. Для гэтага выразім $\frac{\partial ^2 u}{\partial ^2 x}$ з зыходнага ўраўнення: $\frac{\partial ^2 u}{\partial ^2 x} = \frac{\partial u}{\partial t} - f(x,t)$, а першую прыватную вытворную па $t$ будзем апраксімаваць пры дапамозе левай рознаснай вытворнай, то бок $\frac{\partial u}{\partial t} \approx u(x,t)_{\overline{t}}.$ \\ \\

Запішам атрыманыя ўмовы ў індэкснай форме:
\begin{equation} \label{eq:heat-system}
	\begin{cases}
	-\frac{\sigma}{h^2}y_{i-1, j+1} + (\frac{1}{\tau} + \frac{2\sigma}{h^2})y_{i, j+1} -\frac{\sigma}{h^2}y_{i+1, j+1} = \frac{y_{ij}}{\tau} + \frac{1-\sigma}{h^2}(y_{i+1, j} -2y_{ij} + y_{i+1, j}) + f _{ij}, \hspace{10pt} i = \overline{1, N-1}\\
	(\frac{\beta _0}{h} - \alpha _0 + \frac{\beta _0 h}{2\tau})y_{0, j+1} - \frac{\beta _0}{h} y_{1, j+1} = -\mu _{0, j+1} + \frac{\beta _0 h}{2 \tau}(y_{0j} + \tau f _{0, j+1}),  \\
	-\frac{\beta _1}{h}y_{N-1, j+1} + (\alpha _1 + \frac{\beta _1}{h} + \frac{\beta _1 h}{2 \tau})y_{N, j+1} = \mu _{N, j+1} + \frac{\beta _1 h}{2 \tau}(y_{Nj} + \tau f_{N, j+1}), \\
	j = \overline{0, N-1}. \\
	\end{cases}
\end{equation}
$y_{i0}, i = \overline{0, N}$ вядомыя з умоваў. Каб атрымаць $y_{i1}$, трэба рашыць сістэму лінейных алгебраічных ураўненняў \eqref{eq:heat-system} адносна $y_{i, j+1}$, матрыца якой мае трохдыяганальны выгляд. Сістэму рашаем метадам прагонкі, па аналагічнай схеме з $y_{i1}$ знаходзім $y_{i2}$, і гэтак далей. \\ \\

\subsection*{Рэалізацыя}

{\small
\begin{verbatim}
# Defining the problem.
f = lambda x, t: x * x - 2 * t
x_range = [0., 1.]
t_range = [0., 1.]
u0 = lambda x: 0.
m0 = lambda t: 0.
m1 = lambda t: 2 * t
a0 = 0
a1 = 0
b0 = 0
b1 = 1
s = 0.5
hx = 0.05
ht = 0.05

nx = int((x_range[1] - x_range[0]) / hx) + 1
nt = int((t_range[1] - t_range[0]) / ht) + 1

def solve_heat_equation():
    grid = [[0. for x in range(nt)] for y in range(nx)]
    for i in range(nx):
        x = x_range[0] + i * hx
        grid[0][i] = u0(x)

    for j in range(1, nt):
        t_temp = t_range[0] + j * ht
        # Running tridiagonal algorithm for each row:
        # Allocating memory.
        matr = [[0. for x in range(nx)] for y in range(nx)]
        rhs = [0. for x in range(nx)]
        for i in range(1, nx-1):
            matr[i][i-1] = -s/(hx*hx)
            matr[i][i] = (1./ht + 2.*s/(hx*hx))
            matr[i][i+1] = -s/(hx*hx)
            x_temp = x_range[0] + i * hx
            rhs[i] = grid[j-1][i] / ht + (1 - s) / (hx * hx) * (grid[j-1][i-1] -
                    2 * grid[j-1][i] + grid[j-1][i+1]) + f(x_temp, t_temp)

        matr[0][0] = b0/hx - a0 + b0*hx/(2*ht)
        matr[0][1] = -b0/hx
        rhs[0] = -m0(t_temp) + b0*hx/(2*ht)*(grid[j-1][0] + ht * f(x_range[0], t_temp-ht))

        matr[nx-1][nx-2] = -b1/hx
        matr[nx-1][nx-1] = a1 + b1/hx + b1*hx/(2*ht)
        rhs[nx-1] = m1(t_temp) + b1*hx/(2*ht) * (grid[j-1][nx-1] + ht * f(x_range[1], t_temp-ht))

        x = tridiagonal_matrix_algo(matr, rhs)
        for i in range(0, len(x)):
            grid[i][j] = x[i]

    for i in range(nx-1, -1, -1):
        for j in range(0, nt):
            value = '{0:.3f} '.format(grid[i][j])
            sys.stdout.write(value)
        sys.stdout.write('\n')
\end{verbatim}
}

\subsection*{Вынік}

Для сеткі на квадраце $1\times1$ з $h = \tau = 0.05$ атрымліваем наступныя значэнні:\par

{\tiny
\begin{verbatim}
-0.0000 0.0047 0.0143 0.0288 0.0481 0.0723 0.1014 0.1354 0.1743 0.2180 0.2666 0.3201 0.3785 0.4418 0.5100 0.5831 0.6610 0.7438 0.8315 0.9241 1.0216
-0.0000 0.0046 0.0138 0.0276 0.0460 0.0691 0.0968 0.1291 0.1661 0.2077 0.2539 0.3048 0.3603 0.4205 0.4853 0.5547 0.6287 0.7074 0.7908 0.8787 0.9714
-0.0000 0.0044 0.0132 0.0264 0.0439 0.0658 0.0922 0.1228 0.1579 0.1974 0.2412 0.2895 0.3421 0.3991 0.4605 0.5263 0.5964 0.6710 0.7500 0.8333 0.9211
-0.0000 0.0043 0.0126 0.0252 0.0418 0.0626 0.0875 0.1165 0.1497 0.1870 0.2285 0.2741 0.3238 0.3777 0.4357 0.4978 0.5641 0.6345 0.7091 0.7878 0.8707
-0.0000 0.0041 0.0121 0.0239 0.0397 0.0593 0.0828 0.1102 0.1415 0.1766 0.2157 0.2586 0.3055 0.3562 0.4108 0.4693 0.5317 0.5980 0.6682 0.7423 0.8203
-0.0000 0.0039 0.0115 0.0227 0.0375 0.0560 0.0781 0.1038 0.1332 0.1662 0.2029 0.2432 0.2871 0.3347 0.3859 0.4408 0.4993 0.5615 0.6273 0.6967 0.7698
-0.0000 0.0038 0.0109 0.0215 0.0354 0.0527 0.0734 0.0974 0.1249 0.1558 0.1900 0.2276 0.2687 0.3131 0.3609 0.4122 0.4668 0.5248 0.5863 0.6511 0.7193
-0.0000 0.0036 0.0103 0.0202 0.0332 0.0493 0.0686 0.0910 0.1166 0.1453 0.1771 0.2121 0.2502 0.2915 0.3359 0.3835 0.4343 0.4881 0.5452 0.6054 0.6688
-0.0000 0.0034 0.0097 0.0189 0.0310 0.0460 0.0638 0.0846 0.1082 0.1347 0.1641 0.1965 0.2317 0.2698 0.3108 0.3548 0.4016 0.4514 0.5041 0.5596 0.6181
-0.0000 0.0033 0.0091 0.0176 0.0288 0.0426 0.0590 0.0780 0.0998 0.1241 0.1511 0.1808 0.2131 0.2481 0.2857 0.3260 0.3689 0.4145 0.4628 0.5138 0.5674
-0.0000 0.0030 0.0085 0.0163 0.0265 0.0391 0.0541 0.0715 0.0913 0.1134 0.1380 0.1650 0.1944 0.2262 0.2604 0.2971 0.3361 0.3776 0.4215 0.4678 0.5165
-0.0000 0.0029 0.0079 0.0150 0.0242 0.0356 0.0492 0.0649 0.0827 0.1027 0.1249 0.1492 0.1757 0.2043 0.2351 0.2681 0.3032 0.3406 0.3801 0.4217 0.4656
-0.0000 0.0026 0.0072 0.0136 0.0219 0.0321 0.0442 0.0582 0.0741 0.0919 0.1116 0.1332 0.1568 0.1823 0.2097 0.2390 0.2702 0.3034 0.3385 0.3755 0.4145
-0.0000 0.0024 0.0065 0.0122 0.0195 0.0285 0.0391 0.0514 0.0654 0.0810 0.0982 0.1172 0.1378 0.1601 0.1841 0.2097 0.2371 0.2661 0.2968 0.3292 0.3633
-0.0000 0.0022 0.0058 0.0107 0.0171 0.0249 0.0340 0.0446 0.0566 0.0700 0.0848 0.1010 0.1187 0.1378 0.1584 0.1803 0.2038 0.2286 0.2550 0.2827 0.3119
-0.0000 0.0020 0.0051 0.0093 0.0146 0.0211 0.0288 0.0376 0.0477 0.0588 0.0712 0.0847 0.0995 0.1154 0.1325 0.1508 0.1703 0.1910 0.2129 0.2361 0.2604
-0.0000 0.0017 0.0042 0.0077 0.0121 0.0174 0.0235 0.0306 0.0386 0.0476 0.0574 0.0683 0.0800 0.0928 0.1064 0.1211 0.1366 0.1532 0.1707 0.1892 0.2087
-0.0000 0.0014 0.0035 0.0061 0.0095 0.0134 0.0181 0.0234 0.0294 0.0361 0.0435 0.0516 0.0604 0.0699 0.0802 0.0911 0.1028 0.1152 0.1283 0.1421 0.1567
-0.0000 0.0010 0.0025 0.0044 0.0067 0.0094 0.0125 0.0161 0.0201 0.0245 0.0294 0.0348 0.0406 0.0469 0.0537 0.0609 0.0687 0.0769 0.0856 0.0948 0.1045
-0.0000 0.0008 0.0017 0.0027 0.0038 0.0051 0.0066 0.0083 0.0103 0.0125 0.0149 0.0175 0.0204 0.0236 0.0269 0.0306 0.0344 0.0385 0.0429 0.0475 0.0523
 0.0000 0.0000 0.0000 0.0000 0.0000 0.0000 0.0000 0.0000 0.0000 0.0000 0.0000 0.0000 0.0000 0.0000 0.0000 0.0000 0.0000 0.0000 0.0000 0.0000 0.0000
\end{verbatim}
}

Метад будаваўся з дакладнасцю ў $\mathcal{O}(h^2)$, таму для кожнага вузла маем дакладнасць прынамсі ў 2 знака пасля коскі.
