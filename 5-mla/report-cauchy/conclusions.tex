\section*{Высновы}

\subsection*{Ураўненне}

Для задачы Кашы \eqref{eq:my-equation} вядомае дакладнае рашэнне:
\begin{equation}
    u(x) = sin(x) - 1
\end{equation}
Таму мы маем магчымасць пабудаваць дакладную сетку значэнняў і параўнаць гэтыя значэнні з вынікамі працы кожнага з алгарытмаў.

{\small
\begin{verbatim}
Exact solution:
y(0.0) = -1.0
y(0.1) = -0.900166583353
y(0.2) = -0.801330669205
y(0.3) = -0.704479793339
y(0.4) = -0.610581657691
y(0.5) = -0.520574461396
y(0.6) = -0.435357526605
y(0.7) = -0.355782312762
y(0.8) = -0.2826439091
y(0.9) = -0.216673090373
y(1.0) = -0.158529015192
\end{verbatim}
}

Знойдзем хібнасці кожнага з метадаў, то бок для кожнага $x_i$ з сеткі знойдзем розніцу $u_i - y_i = u(x_i) - y(x_i)$, дзе $u_i$ - дакладнае значэнне, $y_i$ - набліжэнне.\par
\vspace{5mm}
Метад шэрагаў:
{\small
\begin{verbatim}
Series method
u(0.0) - y(0.0) = 0.0
u(0.1) - y(0.1) = 8.33134948808e-08
u(0.2) - y(0.2) = 5.74204981008e-07
u(0.3) - y(0.3) = 1.43037073175e-06
u(0.4) - y(0.4) = 2.6126972027e-06
u(0.5) - y(0.5) = 4.08658620521e-06
u(0.6) - y(0.6) = 5.82281987083e-06
u(0.7) - y(0.7) = 7.798065847e-06
u(0.8) - y(0.8) = 9.99512268285e-06
u(0.9) - y(0.9) = 1.24029964328e-05
u(1.0) - y(1.0) = 1.50168860044e-05
\end{verbatim}
}

Яўны метад Эйлера:
{\small
\begin{verbatim}
Explicit Euler:
u(0.0) - y(0.0) = 0.0
u(0.1) - y(0.1) = -0.000166583353172
u(0.2) - y(0.2) = -0.000814510619714
u(0.3) - y(0.3) = -0.00189046507399
u(0.4) - y(0.4) = -0.00334537531258
u(0.5) - y(0.5) = -0.00513614894625
u(0.6) - y(0.6) = -0.0072267308694
u(0.7) - y(0.7) = -0.00958863068112
u(0.8) - y(0.8) = -0.0122010668214
u(0.9) - y(0.9) = -0.0150508625166
u(1.0) - y(1.0) = -0.0181322075461
\end{verbatim}
}

Няяўны метад Эйлера:
{\small
\begin{verbatim}
Implicit Euler:
u(0.0) - y(0.0) = 0.0
u(0.1) - y(0.1) = 0.000302864932127
u(0.2) - y(0.2) = 0.00103106960983
u(0.3) - y(0.3) = 0.00214351842673
u(0.4) - y(0.4) = 0.0036036376648
u(0.5) - y(0.5) = 0.00538040298753
u(0.6) - y(0.6) = 0.00744898502389
u(0.7) - y(0.7) = 0.00979111440253
u(0.8) - y(0.8) = 0.0123952609807
u(0.9) - y(0.9) = 0.0152567105458
u(1.0) - y(1.0) = 0.0183776087035
\end{verbatim}
}

Метад паслядоўнага павышэння парадка дакладнасці 3-га парадка:
{\small
\begin{verbatim}
Predictor-corrector:
u(0.0) - y(0.0) = 0.0
u(0.1) - y(0.1) = 1.01615236159e-06
u(0.2) - y(0.2) = 1.81446849989e-06
u(0.3) - y(0.3) = 2.40781471561e-06
u(0.4) - y(0.4) = 2.81657031409e-06
u(0.5) - y(0.5) = 3.06641340897e-06
u(0.6) - y(0.6) = 3.18608459438e-06
u(0.7) - y(0.7) = 3.20528264675e-06
u(0.8) - y(0.8) = 3.15281398755e-06
u(0.9) - y(0.9) = 3.05508399595e-06
u(1.0) - y(1.0) = 2.93498577661e-06
\end{verbatim}
}

Метад Рунге-Кутта:
{\small
\begin{verbatim}
Runge-Kutta:
u(0.0) - y(0.0) = 0.0
u(0.1) - y(0.1) = -3.9528923035e-05
u(0.2) - y(0.2) = -6.26226403999e-05
u(0.3) - y(0.3) = -7.11839729668e-05
u(0.4) - y(0.4) = -6.73895820182e-05
u(0.5) - y(0.5) = -5.35650719187e-05
u(0.6) - y(0.6) = -3.20782258875e-05
u(0.7) - y(0.7) = -5.25113498062e-06
u(0.8) - y(0.8) = 2.47094983472e-05
u(0.9) - y(0.9) = 5.57655270245e-05
u(1.0) - y(1.0) = 8.60861279295e-05
\end{verbatim}
}

Экстрапаляцыйны метад Адамса 3-га парадка:
{\small
\begin{verbatim}
Extra Adams:
u(0.0) - y(0.0) = 0.0
u(0.1) - y(0.1) = 1.01615236159e-06
u(0.2) - y(0.2) = 1.81446849989e-06
u(0.3) - y(0.3) = 6.94558823411e-06
u(0.4) - y(0.4) = 1.48808438898e-05
u(0.5) - y(0.5) = 2.56613680059e-05
u(0.6) - y(0.6) = 3.89519706653e-05
u(0.7) - y(0.7) = 5.45762439096e-05
u(0.8) - y(0.8) = 7.23834555442e-05
u(0.9) - y(0.9) = 9.2284260969e-05
u(1.0) - y(1.0) = 0.000114242122081
\end{verbatim}
}

Такім чынам, найлепшыя набліжэнні былі атрыманыя метадамі паслядоўнага павышэння дакладнасці 3-га парадку і метадам Рунге-Кутта - абодва далі 4-6 знакаў пасля коскі, якія супалі з дакладным рашэннем. Разам з тым экстрапаляцыйны метад Адамса, які таксама мусіць даваць 3-ці парадак, даў горшыя вынікі за два вышэй згаданых падыхода. Метады Эйлера далі вынікі, якія і чакаліся, якія, аднак, на практыцы маюць меншую вартасць за вынікі апошніх метадаў. Няяўны метад Эйлера, да таго ж, адносна складаны ў рэалізацыі. У сваю чаргу, метады Рунге-Кутта і прэдыктар-карэктар простыя і прыемныя ў рэалізацыі і даюць пажаданыя вынікі.

\subsection*{Сістэма}
У адрозненні ад ўраўнення, дакладнага рашэння для сістэмы ўраўненняў мы не маем, таму вынікі можам рабіць толькі на падставе параўнання вынікаў працы алгарытмаў паміж сабой. Гэта кепскі падыход, які не дае дакладных лічбаў хібнасці падлікаў і па якім нават нельга меркаваць пра слушнасць знойдзенага рашэння. Разам з тым заўважым, што раскід значэнняў вельмі малы і рэдка дасягае значэння ў $0.1$. Пры параўнанні вынікаў метадаў вышэйшых парадкаў якія, як мяркуецца, павінны даваць бліжэйшы да дакладнага рашэння вынік, можна заўважыць, што розніца ў значэннях з'яўляецца толькі ў 3 ці 4 знаку пасля коскі (напрыклад, пры параўнанні інтэрпаляцыйнага метада Адамса і метада Рунге-Кутта).\par
\vspace{10mm}
Як і для ўраўнення, так і для сістэмы найбольш надзейнымі апынуліся метады Рунге-Кутта, а таксама паслядоўнага павышэння парадка дакладнасці (прэдыктар-карэктар).
