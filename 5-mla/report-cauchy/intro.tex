Перад намі ставіцца задача набліжанымі метадамі на адрэзку $[x_0, x_1]$ рашыць задачу Кашы для АДУ-1:
\begin{equation}
    \begin{cases}
        u^{\prime} = f(x,u), \hspace{10pt} x \in [x_0, x_1],\\
        y(x_0) = u_0. \\
    \end{cases}
\end{equation}
і задачу Кашы для сістэмы АДУ-1:
\begin{equation}
    \begin{cases}
        u^{\prime} = f(x, u, v), \\
        v^{\prime} = g(x, u, v), \hspace{10pt} x \in [x_0, x_1],\\
        u(x_0) = u_0, \hspace{10pt} v(x_0) = v_0. \\
    \end{cases}
\end{equation} \\
Вынікам працы кожнага алгарытма будзе разбіццё адрэзка $[x_0, x_1]$ на $N$ частак і падлік у адпаведных кропках набліжаных значэнняў функцыі $u(x)$ альбо сістэмы функцый $u(x), v(x)$.\par
\vspace{5mm}
Да ўраўнення прыменім метад шэрагаў трэцяга парадка дакладнасці, яўны і няяўны метады Эйлера, метад паслядоўнага павышэння парадка дакладнасці (прэдыктар-карэктар) трэцяга парадка дакладнасці, метад Рунге-Кутта другога парадка дакладнасці, экстрапаляцыйны метад Адамса другога парадка.\par
\vspace{5mm}
Да сістэмы ўраўненняў будуць прымененыя яўны і няяўны метады Эйлера, метад паслядоўнага павышэння парадка дакладнасці другога парадка, метад Рунге-Кутта другога парадка дакладнасці, інтэрпаляцыйны метад Адамса трэцяга парадка дакладнасці.\par

\subsection*{Умовы канкрэтнай задачы}
Ураўненне:\\
\begin{equation} \label{eq:my-equation}
    \begin{cases}
        u^{\prime} = -y cos(x) + sin(x)cos(x), \hspace{10pt} x \in [0, 1],\\
        y(0) = -1 \\
    \end{cases}
\end{equation}
Сістэма ўраўненняў:
\begin{equation} \label{eq:my-system}
    \begin{cases}
        y^{\prime} = e ^{-(y^2+z^2)} + 2x, \\
        z^{\prime} = 2y^2 + z, \hspace{10pt} x \in [0, 0.5], \\
        y(0) = 0.5, \\
        z(0) = 1. \\
    \end{cases}
\end{equation}

\subsection*{Вызначэнне ў кодзе}
Ураўненне:
{\small
\begin{verbatim}
# Defining the Cauchy problem for equation.
f = lambda x, u: - u * math.cos(x) + math.sin(x) * math.cos(x)
x_range = [0., 1.]
x0 = 0.
u0 = -1.

du_dx = f

# Defining a grid to make calculations on.
n = 10
h = (x_range[1] - x_range[0]) / n
grid = [x_range[0] + i * h for i in range(n+1)]

# Exact solution of the problem.
exact_solution = lambda x: -1 + math.sin(x)

# Some pre-calculations. k is for highest degree of derivations.
k = 3
d_dx = [
    lambda x, u: u0,
    lambda x, u: du_dx(x, u),
    lambda x, u: u * math.sin(x) + math.cos(2 * x) + f(x, u) * (- math.cos(x)),
    lambda x, u: u * math.cos(x) - 2 * math.sin(2 * x) + 2 * math.sin(x) * f(x, u) -
            math.cos(x) * d_dx[2](x, u),
]
\end{verbatim}
}

Сістэма ўраўненняў:
{\small
\begin{verbatim}
# Defining the problem (system of differential equations).
dy_dx = lambda x, y, z: math.e**(-(y*y + z*z)) + 2 * x
dz_dx = lambda x, y, z: 2 * y * y + z
x_range = [0., 0.5]
x0 = 0
y0 = 0.5
z0 = 1.

f = dy_dx
g = dz_dx

# Defining the grid to make the calculations.
n = 10
h = (x_range[1] - x_range[0]) / n
grid = []
for i in range(0, n):
    grid.append(x_range[0] + i * h)
\end{verbatim}
}
