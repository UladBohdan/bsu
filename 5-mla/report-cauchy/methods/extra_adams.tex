\section{Экстрапаляцыйны метад Адамса}
Будзем будаваць экстрапаляцыйны метад Адамса 3-га парадка дакладнасці, то бок $q=2$. $A_i, i = \overline{0, 2}$ знаходзім з сістэмы:
\begin{equation}
    \sum _{i = 0}^{2} A_i (-i)^j = \frac{1}{j+1} , \hspace{10pt} j = \overline{0,2}
\end{equation}
Атрымліваем: $A_0 = \frac{23}{12}$, $A_1 = -\frac{16}{12}$, $A_2 = \frac{5}{12}$.\\
Тады $y_{n+1}$ падлічваецца па формуле:
\begin{equation}
    y_{n+1} = y_n + \frac{h}{12}(23f_n - 16f_{n-1} + 5f_{n-2}), \hspace{15pt} h = \frac{x_1 - x_0}{N},\hspace{15pt} n = \overline{2, N-1}
\end{equation}
Патрэбныя для запуску алгарытма $y_1$, $y_2$ возьмем з метада паслядоўнага павышэння дакладнасці 3-га парадка дакладнасці.

{\small
\begin{verbatim}
def extra_adams(approx):
    yi = approx
    for i in range(3, len(grid)):
        step = grid[i] - grid[i-1]
        y_new = yi[i-1] + step * (23 * f(grid[i-1], yi[i-1]) - 15 * f(grid[i-2], yi[i-2]) +
                5 * f(grid[i-3], yi[i-3])) / 12.
        yi.append(y_new)
    return yi
\end{verbatim}
}

{\small
\begin{verbatim}
Extra Adams:
y(0.0) = -1.0
y(0.1) = -0.900167599506
y(0.2) = -0.801332483673
y(0.3) = -0.704486738927
y(0.4) = -0.610596538535
y(0.5) = -0.520600122764
y(0.6) = -0.435396478576
y(0.7) = -0.355836889006
y(0.8) = -0.282716292556
y(0.9) = -0.216765374633
y(1.0) = -0.158643257314
\end{verbatim}
}
