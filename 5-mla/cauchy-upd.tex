\documentclass{article}

\usepackage{geometry}
 \geometry{
 a4paper,
 total={170mm,257mm},
 left=20mm,
 top=20mm,
 }

\usepackage[utf8]{inputenc}
\usepackage[russian]{babel}

\usepackage{amsfonts}
\usepackage{amsmath}
\usepackage{verbatim}

\setlength{\parindent}{0pt}

\begin{document}

Дадатак да лабараторнай работы па набліжаным рашэнні задачы Кашы для АДУ-1.
\section*{Метад Шэрагаў}
Пералічыў вытворныя і выправіў дробную памылку ў алгарытме. Новыя вынікі:
{\small
\begin{verbatim}
Series method:
y(0.0) = -1.0
y(0.1) = -0.900166666667
y(0.2) = -0.80133124341
y(0.3) = -0.704481223709
y(0.4) = -0.610584270389
y(0.5) = -0.520578547982
y(0.6) = -0.435363349425
y(0.7) = -0.355790110828
y(0.8) = -0.282653904223
y(0.9) = -0.216685493369
y(1.0) = -0.158544032078
\end{verbatim}
}
Розніца паміж дакладнымі значэннямі і значэннямі з метада шэрагаў:
{\small
\begin{verbatim}
Series method
u(0.0) - y(0.0) = 0.0
u(0.1) - y(0.1) = 8.33134948808e-08
u(0.2) - y(0.2) = 5.74204981008e-07
u(0.3) - y(0.3) = 1.43037073175e-06
u(0.4) - y(0.4) = 2.6126972027e-06
u(0.5) - y(0.5) = 4.08658620521e-06
u(0.6) - y(0.6) = 5.82281987083e-06
u(0.7) - y(0.7) = 7.798065847e-06
u(0.8) - y(0.8) = 9.99512268285e-06
u(0.9) - y(0.9) = 1.24029964328e-05
u(1.0) - y(1.0) = 1.50168860044e-05
\end{verbatim}
}
Цяпер падобна на праўду. $\mathcal{O}(h^4)$ выконваецца.

\section*{Яўны метад Эйлера для ўраўнення}
Выправіў адзін хібны індэкс ў рэалізацыі метада. Новыя вынікі:
{\small
\begin{verbatim}
Explicit Euler:
y(0.0) = -1.0
y(0.1) = -0.9
y(0.2) = -0.800516158585
y(0.3) = -0.702589328265
y(0.4) = -0.607236282379
y(0.5) = -0.51543831245
y(0.6) = -0.428130795736
y(0.7) = -0.346193682081
y(0.8) = -0.270442842279
y(0.9) = -0.201622227856
y(1.0) = -0.140396807646
\end{verbatim}
}
Розніца паміж дакладным значэннем і значэннямі, атрыманымі яўным метадам Эйлера:
{\small
\begin{verbatim}
Explicit Euler:
u(0.0) - y(0.0) = 0.0
u(0.1) - y(0.1) = -0.000166583353172
u(0.2) - y(0.2) = -0.000814510619714
u(0.3) - y(0.3) = -0.00189046507399
u(0.4) - y(0.4) = -0.00334537531258
u(0.5) - y(0.5) = -0.00513614894625
u(0.6) - y(0.6) = -0.0072267308694
u(0.7) - y(0.7) = -0.00958863068112
u(0.8) - y(0.8) = -0.0122010668214
u(0.9) - y(0.9) = -0.0150508625166
u(1.0) - y(1.0) = -0.0181322075461
\end{verbatim}
}
Вынікі для няяўнага метада Эйлера застаюцца папярэднія. Заўважым, што хібнасці яўнага і няяўнага метадаў адрозніваюцца знакам, і ў астатнім паводзяць сябе падобна - візуальна значэнні яўнага і няяўнага метадаў размешчаныя сіметрычна адносна дакладнага рашэння.

\section*{Экстрапаляцыйны метад Адамса 3-га парадка для ўраўнення}
Выправіў дробную адрукоўку ў алгарытме: $A_1 = - \frac{16}{12}$ замест $A_1 = - \frac{15}{12}$.
Новыя вынікі:
{\small
\begin{verbatim}
Extra Adams:
y(0.0) = -1.0
y(0.1) = -0.900167599506
y(0.2) = -0.801332483673
y(0.3) = -0.704486738927
y(0.4) = -0.610596538535
y(0.5) = -0.520600122764
y(0.6) = -0.435396478576
y(0.7) = -0.355836889006
y(0.8) = -0.282716292556
y(0.9) = -0.216765374633
y(1.0) = -0.158643257314
\end{verbatim}
}
Розніца дакладных значэнняў са значэннямі з экстрапаляцыйнага метада Адамса 3-га парадка.
{\small
\begin{verbatim}
Extra Adams:
u(0.0) - y(0.0) = 0.0
u(0.1) - y(0.1) = 1.01615236159e-06
u(0.2) - y(0.2) = 1.81446849989e-06
u(0.3) - y(0.3) = 6.94558823411e-06
u(0.4) - y(0.4) = 1.48808438898e-05
u(0.5) - y(0.5) = 2.56613680059e-05
u(0.6) - y(0.6) = 3.89519706653e-05
u(0.7) - y(0.7) = 5.45762439096e-05
u(0.8) - y(0.8) = 7.23834555442e-05
u(0.9) - y(0.9) = 9.2284260969e-05
u(1.0) - y(1.0) = 0.000114242122081
\end{verbatim}
}

\end{document}
