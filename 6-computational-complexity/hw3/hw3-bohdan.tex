\documentclass{article}

\usepackage{geometry}
 \geometry{
 a4paper,
 total={170mm,257mm},
 left=20mm,
 top=20mm,
 }

\usepackage[utf8]{inputenc}
\usepackage[russian]{babel}

\usepackage{amsfonts}
\usepackage{amsmath}

\usepackage[none]{hyphenat}

\usepackage{color}

\setlength{\parindent}{0pt}
\setlength{\parskip}{1em}

\newcommand{\opt}{${\text{П}}_{opt}\ $}

\begin{document}

{\large

Богдан Уладзіслаў

ФПМІ, 3 курс, 3 група

\vspace{5mm}

ДЗ 3

NP-складанасць задачы $1||\sum w_j u_j$

}

\vspace{15mm}

Будзем даказываць NP-паўнату адпаведнай задачы распазнавання - з доказу будзе вынікаць і NP-складанасць нашай задачы.
Задача распазнавання фармулюецца наступным чынам: для кожнай работы зададзеныя $p_j$, $d_j$, $w_j$, трэба пабудаваць
расклад для адной машыны, пры якім $\sum w_j u_j < y = const$.

Доказ будзем праводзіць, выкарыстоўваючы эталонную задачу РАЗБІЦЦЁ ($\sum a_i = 2A$, ці існуе разбіццё $a_i$ на два мноствы,
кожнае з якіх дае $A$ у суме). Пакажам, што калі задача РАЗБІЦЦЁ мае рашэнне, то
будзем мець ТАК-прыклад у нашай задачы распазнавання і калі задача РАЗБІЦЦЁ рашэння не мае, то будзе мець НЕ-прыклад.

Прыклад нашай задачы распазнавання будуецца па наступных правілах: $p_j = a_j, w_j = a_j, d_j = A, y = A$ (будуецца за лінейны час).

1. Няхай задача РАЗБІЦЦЁ мае рашэнне. Рашэнне існуе - таму ёсць разбіццё на мноствы $E_1$ i $E_2$. Не губляючы агульнасці,
усе патрабаванні з $E_1$ паспеюць выканацца
да дырэктыўнага тэрміну, таму адпаведныя $u_j = 0$; патрабаванні з $E_2$ да дырэктыўнага тэрміну выканацца не паспеюць - будзем мець
штраф $\sum w_j u_j = \sum a_j u_j = A \le y$. Маем ТАК-прыклад задачы распазнавання.

2. Няхай задача РАЗБІЦЦЁ не мае рашэння - не існуе разбіцця на два мноствы з аднолькавай сумай роўнай $A$. Як і ў папярэднім пункце -
ставім перагародку ў момант часу $A$ (у папярэднім пункце гэтая перагародка ніколі не перашкаджала, бо адно з патрабаванняў сканчвалася
у момант часу $A$, а наступнае ў гэты ж момант пачыналася). Адсутнасць рашэння задачы РАЗБІЦЦЁ гарантуе, што перагародка перашкодзіць
аднаму з патрабаванняў пачацца ў моманту часу $<A$ і скончыцца ў момант часу $>A$. Будзем мець прастой машыны (мінімум у адну часавую адзінку)
непасрэдна перад момантам $A$. Гэта, у сваю чаргу, гарантуе, што патрабаванні, якія пачнуцца ў момант часу $A$ і пазней, у суме
па часе будуць $>A$. Улічваючы правілы будування прыклада, маем: $\sum w_j u_j = $[$u_j = 0$ для ўсіх патрабаванняў, што пачаліся і скончыліся
раней за $A$.] $=\sum a_j u_j > A = y$. Маем НЕ-прыклад задачы распазнавання.

Паказалі палінаміальную прывадзімасць задачы РАЗБІЦЦЯ да нашай задачы распазнавання, таксама задача распазнавання прыналежыць класу NP
(бо пры наяўнасці прыклада (сертыфіката) праверка, ці гэта ТАК- ці НЕ-прыклад, здзяйсняецца за палінаміальны час) -
таму нашая задача распазнавання NP-поўная.
Адсюль вынікае NP-складанасць пачатковай задачы аптымізацыі. Што і трэба было даказаць.

\end{document}
