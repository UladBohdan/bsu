\documentclass{article}

\usepackage{geometry}
 \geometry{
 a4paper,
 total={170mm,257mm},
 left=20mm,
 top=20mm,
 }

\usepackage[utf8]{inputenc}
\usepackage[russian]{babel}

\usepackage{amsfonts}
\usepackage{amsmath}

\usepackage[none]{hyphenat}

\setlength{\parindent}{0pt}
\setlength{\parskip}{1em}

\newcommand{\opt}{${\text{П}}_{opt}\ $}

\begin{document}

{\large

Богдан Уладзіслаў

ФПМІ, 3 курс, 3 група

\vspace{5mm}

ДЗ 2

Аптымізацыйная задача пра разбіццё параў

}

\vspace{15mm}

Будзем даказываць NP-складанасць задачы.

\subsection*{Схема доказу}

\[
{\text{П}}_{1} \propto {\text{П}}_{2} \propto {\text{П}}_{3} \propto {\text{П}}_{4} \propto_{T} {\text{П}}_{opt}
\]

дзе:

${\text{П}}_{1}$ - Задача пра разбіццё: мноства натуральных лікаў $e_k$ разбіваецца на два
падмноства $E_1$ і $E_2$ з роўнымі сумамі элементаў. Ведаем пра NP-паўнату
задачы.

${\text{П}}_{2}$ - Задача пра разбіццё параў: элементы $(e_1, e_2), ... (e_{n-1}, e_n)$
размяркоўваюцца паміж двумя падмноствамі з роўнымі сумамі элементаў; элементы з
адной пары прыналежаць розным мноствам.

${\text{П}}_{3}$ - Задача пра разбіццё ўпарадкаваных параў: дадаткова накладваецца
абмежаванне $0 < e_1 < e_2 < ... < e_{n-1} < e_n$. Па-ранейшаму, шукаем такое
разбіццё, што:

\[
F(E_1, E_2) = \sum_{e_k \in E_1}{e_k} - \sum_{e_k \in E_2}{e_k} = 0
\]

${\text{П}}_{4}$ - Задача пра разбіццё ўпарадкаваных параў з мадыфікаванай мэтавай
функцыяй:

\[
F(E_1, E_2) = \sum_{e_k \in E_1}{e_k} - \sum_{e_k \in E_2}{e_k} < y
\]

дзе $y = const > 0$.

\opt - Аптымізацыйная задача пра разбіццё ўпарадкаваных параў.

\subsection*{${\text{П}}_{1} \propto {\text{П}}_{2}$}

Пакажам палінаміальную прывадзімасць. Будзем дзейнічаць наступным чынам:
кожнаму $e_k$ паставім у адпаведнасць пару лікаў $(1, e_k+1)$. Атрымаем мноства:
\[
    (1, e_1+1), (1, e_2+1), ... , (1, e_{n-1}+1), (1, e_n+1)
\]
У задачы ${\text{П}}_{1}$ было знойдзенае разбіццё. Для кожнай новаўтворанай
пары элемент $e_k+1$ размяшчаем у тым з мностваў $E_1, E_2$, у якім дагэтуль
знаходзіўся элемент $e_k$; 1-ку размяшчаем у іншым мностве. Зразумела, што
значэнне функцыі $F(E_1, E_2)$ не зменіцца.

Выснова: задача ${\text{П}}_{2}$ - NP-поўная.

\subsection*{${\text{П}}_{2} \propto {\text{П}}_{3}$}

Пакажам палінаміальную прывадзімасць. Увядзем абазначэнне $A = \sum{e_k}$.
Наступным чынам мадыфікуем выпісаныя ў папярэднім пункце пары:
\[
    (1+A, e_1+1+2A), (1+2A, e_2+1+2A), ... , (1+(n-1)A, e_{n-1}+1+(n-1)A), (1+nA, e_n+1+nA)
\]
Відавочна, што цяпер мы маем строгую ўпарадкаванасць усіх элементаў у шэрагу.

Выснова: задача ${\text{П}}_{3}$ - NP-поўная.

\subsection*{${\text{П}}_{3} \propto {\text{П}}_{4}$}

Задача ${\text{П}}_{3}$ - падзадача задачы ${\text{П}}_{4}$ з $y = 0$. Мы можам
казаць пра NP-паўнату задачы ${\text{П}}_{4}$ таму, што уваходы задачаў ${\text{П}}_{3}$
і ${\text{П}}_{4}$ палінаміальна звязаныя: неглядзячы на адсутнасць яўнага
задання значэння $y$ мы маем яшчэ $n$ лікаў, якія з'яўляюцца ўваходнымі дадзенымі
задачы. Робім выснову пра NP-паўнату задачы ${\text{П}}_{4}$.

\subsection*{${\text{П}}_{4} \propto_{T} {\text{П}}_{opt}$}

NP-складанасць задачы \opt будзем даказываць выкарыстоўваючы прывадзімасць па
Цюрынгу ад адпаведнай задачы распазнавання ${\text{П}}_{4}$
(фармулюецца аналігачна задачы аптымізацыі,
умова мінімізацыі замяняецца ўмовай $F(E_1, E_2) < y$, для зададзенага $y$).

Па Тэарэме 3.4 з NP-паўнаты вынікае NP-складанасць задачы ${\text{П}}_{4}$
(задача ${\text{П}}_{4}$ - NP-складаная).

Мяркуем справядлівасць гіпотэзы пра несупадзенне класаў P і NP.

Тады праз стандартную схему доказу NP-складанасці аптымізацыйнай задачы
(апісанай у тэкстах лекцый) для доказу NP-складанасці задачы \opt застаецца
паказаць палінаміальную вылічальнасць функцыі $F(I, x^{*})$. Тут $I \in D_{\text{П}}$ -
прыклад задачы, $x^{*} \in X(I)$ - элемент з канечнага мноства дапускальных
элементаў для прыклада $I$.

\[
F(I, x^{*}) = F(E_1, E_2) = \sum_{e_k \in E_1}{e_k} - \sum_{e_k \in E_2}{e_k}
\]

Зразумела, што за палінаміальны час мы можам праверыць слушнасць сцверджання
$F(I, x^{*}) < y$ пры наяўнасці $I$(апісвае задачу, то бок мноства элементаў
$e_k$), $x^{*}$ (які задае разбіццё паміж мноствамі $E_1$ і $E_2$) і $y$. Падлік
дзвюх сумаў элементаў здзяйсняецца за лінейны адносна колькасці элементаў час,
то бок за палінаміальны адносна памеру ўваходных дадзеных.

Такім чынам, мы паказалі NP-складанасць аптымізацыйнай задачы пра разбіццё параў.

\end{document}
