\documentclass{article}

\usepackage{geometry}
 \geometry{
 a4paper,
 total={170mm,257mm},
 left=20mm,
 top=20mm,
 }

\usepackage[utf8]{inputenc}
\usepackage[russian]{babel}

\usepackage{amsfonts}
\usepackage{amsmath}

\usepackage[none]{hyphenat}

\setlength{\parindent}{0pt}
\setlength{\parskip}{1em}

\newcommand{\opt}{${\text{П}}_{opt}\ $}
\newcommand{\rec}{${\text{П}}^\prime\ $}

\begin{document}

{\large

Богдан Уладзіслаў

ФПМІ, 3 курс, 3 група

\vspace{5mm}

ДЗ 2

Аптымізацыйная задача пра разбіццё параў

}

\vspace{10mm}

Маем аптымізацыйную задачу \opt (ва ўмовах задачы). Праз \rec абазначым
адпаведную задачу распазнавання (фармулюецца аналігачна задачы аптымізацыі,
умова мінімізацыі замяняецца ўмовай $F(E_1, E_2) < y$, для зададзенага $y$).

NP-паўната задачы \rec даказывалася на Кантрольнай рабоце па адпаведнай тэме,
спашлемся на атрыманыя вынікі.

Па Тэарэме 3.4 з NP-паўнаты вынікае NP-складанасць задачы \rec
(задача \rec - NP-складаная).

Мяркуем справядлівасць гіпотэзы пра несупадзенне класаў P і NP.

Тады праз стандартную схему доказу NP-складанасці аптымізацыйнай задачы
(апісанай у тэкстах лекцый) для доказу NP-складанасці задачы \opt застаецца
паказаць палінаміальную вылічальнасць функцыі $F(I, x^{*})$. Тут $I \in D_{\text{П}}$ -
прыклад задачы, $x^{*} \in X(I)$ - элемент з канечнага мноства дапускальных
элементаў для прыклада $I$.

\[
F(E_1, E_2) = \sum_{e_k \in E_1}{e_k} - \sum_{e_k \in E_2}{e_k}
\]

Зразумела, што за палінаміальны час мы можам праверыць слушнасць сцверджання
$F(I, x^{*}) < y$ пры наяўнасці $I$(апісвае задачу, то бок мноства элементаў
$e_k$), $x^{*}$ (які задае разбіццё паміж мноствамі $E_1$ і $E_2$) і $y$. Падлік
дзвюх сумаў элементаў здзяйсняецца за лінейны адносна колькасці элементаў час,
то бок за палінаміальны адносна памеру ўваходных дадзеных.

Такім чынам, мы паказалі NP-складанасць аптымізацыйнай задачы пра разбіццё параў.

\end{document}
